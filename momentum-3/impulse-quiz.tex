\documentclass[quiz,addpoints,noanswers]{exam}

\usepackage[quiz]{../physics9}

\title{Quiz: Momentum -- Impulse!}
\date{\today}
\author{\mobeardInstructorShort}

\begin{document}
\maketitle

\begin{questions}
\question[1] What is the formula for impulse? Please check all that apply. 
\begin{choices}
\CorrectChoice $\text{impulse} = \Delta p$
\CorrectChoice $\text{impulse} = F \Delta t$
\CorrectChoice $\text{impulse} = m \Delta v$
\choice $\text{impulse} = \frac{1}{2} mv^2$
\end{choices}

\question[1] Consider the equation $\Delta p = F \Delta t$. For a given change in momentum $\Delta p$, how can I reduce the average force $F$? 
\begin{solution}[1in]
You can try to spread out the collision over time, increasing $\Delta t$ and decreasing $F$. 
\end{solution}

\question[1] Thought experiment: If you were sitting at rest on a rolling chair and threw a \SI{25}{\pound} medicine ball, what would happen? Please draw a diagram or explain in words. 
\begin{solution}[1in]
Because the initial momentum of the system is zero, you will feel a recoil or reaction force in the opposite direction, giving you a small velocity in the direction opposite the medicine ball depending on the mass of you and the chair. 
\end{solution}

\question[1] When doing a silly mascot trick at a hockey game, Gritty uses a cue stick to hit a hockey puck straight on at another hockey puck of identical mass. Assuming a perfectly elastic collision and that the pucks hit straight on, which of the following is most likely TRUE?
\begin{choices}
\choice Both pucks are stopped ($v=0$) at the end.
\choice The first puck bounces off the second and is moving towards Gritty at the end. 
\CorrectChoice The momentum of the first puck is transferred to the second puck, which continues moving at the end. The first puck stops. 
\choice Do not have enough information to determine the answer. 
\end{choices}

\question[1] As a further promotional gag, the Flyers debut an ``anti-Gritty'' with the same mass (both are \SI{200}{\kilo\gram}).  Gritty and anti-Gritty face each other down, then begin skating towards one another at \SI{8.9}{\meter\per\second} for the final battle (there can be only one).  What is the total momentum of the Gritty-anti-Gritty pair? 
\begin{solution}[1in]
Since their masses are equal $m=\SI{200}{\kilo\gram}$ and their velocities are equal but opposite ($v_{ag} = - v_{g}$) the total momentum of the Gritty-anti-Gritty pair is zero (\SI{0}{\kilo\gram\meter\per\second}). 
\end{solution}
\end{questions}
\end{document}
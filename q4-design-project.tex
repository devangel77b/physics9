\documentclass[hw]{exam}

\usepackage[hw]{physics9}
\title{Project \#3: Not Your Middle School Egg Drop Challenge!}
\date{\printdate{5/3/2021}}
\duedate{\printdate{5/13/2021}}
\author{\mobeardInstructorShort}

\begin{document}
\maketitle

\begin{abstract}
The goal of this project is to design a structure which will reduce the force necessary to stop a falling object.  Since the impulse between the ground and a falling object is fixed by the speed of the object before impact, you must increase the contact time between the object and the ground thus reducing the force required to stop the object. In order to complete this challenge, you will be given some limited materials. A secondary goal of this project is to learn to think like an engineer. 
\end{abstract}

\section{Introduction}
As we have seen in class, impulse is the change in momentum caused by a force in an interaction. The impulse is determined by multiplying the force acting on an object by the time it is in contact with the object. For example, the impulse of a racket on a tennis ball is what rapidly changes the ball’s direction in the span of a fraction of a second. Since the energy change is very large -- remember large velocities means large energies - this change has to be done with either a very large force over a very short time (such as a ball hitting the racket) or a small force over a very long time -- such as air resistance slowly stopping the motion of the ball. The force comes from the motion of the racket pushing against the ball, while the time is determined by how long the ball remains in contact with the racket. 
 
\section{Objectives}
Your device should minimize the force experienced by an egg surrogate (bocce ball) during a drop from \SI{1}{\meter} above the top surface of your device. Force will be measured using a force plate and bucket arrangement that your instructor will demonstrate. 
 
\section{Constraints}
Your design is subject to the following constraints:
\begin{enumerate}
\item Designs are limited to seven (7) of the materials listed on the list of materials: \url{https://docs.google.com/document/d/1aZJONCBz3iWYO8fqYVitu0BGQQ5Fm_ISXR6YYAGMz4o/edit}
\item Device needs to stand on its own. 
\item The device and all materials must start intact. (i.e., you can pick it as a single object).
\item Your structure should allow easy insertion and removal from bucket.
\item Your structure will be assessed based on its design, neatness as well as its ability to minimize impact force. 
\end{enumerate}

\section{Test and evaluation during development} 
The bocce ball, force plate and bucket will be available at all times. During development, you may test your device as much as necessary. 

On the actual Drop Day, a bocce ball will be dropped from a height of \SI{1}{\meter} above the protection device.  You will get \textbf{one chance} to do this correctly on Drop Day. 
 
\section{Deliverables}
\begin{questions}
\question Devices will be tested on \printdate{5/13/2021}. Your device will be evaluated according to the \textbf{Device Rubric} here: \url{https://classroom.google.com/u/0/w/MjQ2MzM0MzA2Njg5/t/all}. 

\question In lieu of homework, you are expected to maintain an Engineering Design Notebook and will complete at least one entry per class period. The Design Notebook records progress on your design.
\begin{parts}
\part A single PDF submitted via Google Classroom.
\part At a \textbf{minimum}, your Design Notebook includes:  A discussion of how your design evolved, including pictures of each of your designs and video of the test drops you performed (be sure to indicate the drop height). This should also include a description of your final device that explains your design (i.e., what precautions have you taken to decrease the impact force and why did you make the choices you made).  Device specs: dimensions, material list, mass - and how did these change throughout the process. Your Design Notebook should include calculations, and all  groups are required to determine the momentum of the ball on impact. Your Design Notebook will be evaluated according to the \textbf{Design Notebook Rubric} here: \url{https://classroom.google.com/u/0/w/MjQ2MzM0MzA2Njg5/t/all}.
\end{parts}
\question Additional engineering deliverables as directed by your instructor, including brainstorming of requirements, concepts, sketches, initial tests, etc. 
\end{questions}


\clearpage
\appendix
\section{Additional background information}
The egg drop project involves several physics concepts that we have studied in class and other concepts that you will have to research. 

\subsection{Momentum}
Momentum is a measure of an object's tendency to move at constant speed along a straight path. Momentum depends on speed and mass. Within a closed system of interacting objects, the total momentum of that system does not change value. This allows one to calculate and predict the outcomes when objects bounce into one another.

When an object is moving, it has a non-zero momentum. If an object is standing still, then its momentum is zero. To calculate the momentum of a moving object multiply the mass of the object times its velocity. Momentum is a vector, which means that momentum is a quantity that has a magnitude, or size, and a direction.
 
\subsection{Pressure}
Pressure is the force per unit area applied to an object in a direction perpendicular to the surface. Pressure is calculated by taking the total force and dividing it by the area over which the force acts. Force and pressure are related but different concepts. A very small pressure, if applied to a large area, can produce a large total force.

\subsection{Air resistance}
A feather and coin will fall with equal accelerations in a vacuum, but unequally in the presence of air. This is because the air molecules cause a frictional force that opposes the motion of the falling objects. This air resistance diminishes the net forces for each. This will be a tiny bit for the coin and very much for the feather. The downward acceleration for the feather is very brief, for the air resistance very quickly builds up to counteract its tiny weight and surface area. The feather does not have to fall very long or very fast for this to happen. When the air resistance of the feather equals the weight of the feather, the net force is zero and no further acceleration occurs.

\subsection{Gravity}
Gravity is a powerful force that has a fundamental impact on the way we live our lives. Even walking, which we take for granted, is not possible without gravity. Gravity provides the necessary downward force on our bodies which creates friction between our feet and the ground, allowing us to walk (push our body weight forward with one leg and then the other).
When other forces are combined with gravity, such as motion (the movement of an object), inertia (the tendency of an object to resist change with regard to movement based on its mass), or power (the ability to exert energy over time), it may be impossible to prevent an impact which will cause damage.

\section{Design concepts for saving the egg}
For instance, if you roll an egg along the ground downhill at considerable velocity towards a wall, you can reasonably expect the egg to break. Your arm provided the force (power) to accelerate the egg to a certain velocity (motion). That motion is being increased due to the acceleration of the egg down the hill (gravity). The egg will not drastically vary its direction and avoid the wall (inertia tends to keep it moving in a straight line). The combination of power, gravity, motion and inertia will probably be sufficient to result in an impact between the egg and the wall that breaks the egg. This impact is called the primary impact.

There is a further impact which takes place when the egg hits the wall; this is when the mass inside the egg impacts against the inside of the wall of the egg. The egg white and egg yolk are usually in liquid form, and though liquid has considerable mass, the liquid inside the egg will rarely be the cause of the egg shell breaking. If you put a steel ball bearing into a plastic egg, and then shake the egg, you can hear the impact of the ball bearing hitting the inside of the egg, and it is easy to imagine the egg cracking because of the steel ball bearing.

The impact resulting from the ball bearing striking the inside of the plastic egg due to the motion or change in motion of the egg is called the secondary impact.

Scientists and engineers have been working for many years to reduce the effect of impacts, primarily in the automobile industry. Efforts to reduce the primary impact (energy absorbing bumpers, crumple zones, modified chassis construction) and efforts to reduce the secondary impact (airbags, padded dashboards, collapsing steering wheels, and seatbelts) are commonplace.

Kinetic energy is the energy that a body possesses as a result of its motion. Potential energy is the energy that exists in a body as a result of its position or condition rather than of its motion.
In building the container, you should think about how the energy is converted from potential energy to kinetic energy, and the work done on the container and the work done on the eggs.
\end{document}




\documentclass [hw,addpoints,noanswers]{exam}

\usepackage[hw]{physics9}

\title{HW: Projectile motion (optional)}
\author{\mobeardInstructorShort}
\date{\today}
\duedate{never}

\begin{document}
\maketitle

\begin{questions}
\question  A cannonball is launched from the top of a \SI{125}{\meter} high cliff with an initial horizontal speed of \SI{20}{\meter\per\second}. The $(x, y)$ coordinate position of the launch location is designated as the $(0, 0)$ position. Determine the $(x, y)$ coordinate positions of the cannonball at \SI{1}{\second} intervals during its path to the ground. Assume $g \sim \SI{10}{\meter\per\second\squared}$, down.
\begin{center}
\includegraphics[width=4in]{hw-projectile-q1.png}
\end{center}
\question A projectile is launched horizontally with a speed of \SI{12.0}{\meter\per\second} from the top of a \SI{24.6}{\meter} high building. Determine the horizontal displacement of the projectile.

\question A pool ball leaves a \SI{0.60}{\meter} high table with an initial horizontal velocity of \SI{2.4}{\meter\per\second}. Predict the time required for the pool ball to fall to the ground and the horizontal distance between the table's edge and the ball's landing location.

\question A soccer ball is kicked horizontally off a \SI{22.0}{\meter} high hill and lands a distance of \SI{35.0}{\meter} from the edge of the hill. Determine the initial horizontal velocity of the soccer ball.



\question  A projectile is launched with an initial speed of \SI{21.8}{\meter\per\second} at an angle of \ang{35.0} above the horizontal\footnote{IM 2/3 - You will need to use trig for this problem}.
\begin{parts}
\part Determine the horizontal and vertical components of the projectile's velocity
\part Determine the peak height of the projectile.
\part Determine the time of flight of the projectile.
\part Determine the horizontal displacement of the projectile.
\end{parts}
\end{questions}
\end{document}
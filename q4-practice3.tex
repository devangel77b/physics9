\documentclass[exam,addpoints, noanswers]{exam}

\usepackage[exam]{physics9}

\title{Practice Test \#4C}
\date{\today}
\author{\mobeardInstructorShort}

\begin{document}
\maketitle
\vfill
\mobeardExamNameBlock
\vfill
Instructions: 
\begin{enumerate}
\item Do not open the exam until the instructor announces that you may begin.
\item Please write your name on each page in case the sheets are separated. 
\item Closed notes, closed book.  You may use two \SI{8.5x11}{\inch} hardcopy note sheets, both sides. You may NOT use your computer or phone as your notes. 
\item Calculators (TI-84 or equivalent) are permitted.  On standardized tests you will not be permitted to use your cell phone as a calculator, so it is best to get used to your calculator now. You may NOT use your computer or phone as your calculator. 
\item Eight multiple choice are worth five points each, three essay questions (divided into parts) are worth 20 points each. 
\item In answering the essay questions, be thorough but concise. Deep understanding of the concepts will be displayed by proper use of vocabulary and discussion of the interconnectedness of concepts. 
\item Show your calculations and (most importantly) your \textbf{thinking}.
\end{enumerate}
\vfill
\begin{center}
\gradetable[h][questions]
\end{center}
\clearpage



\begin{questions}
\question[5] Easy warmup question: what are the correct SI units for energy? 
\begin{choices}
\CorrectChoice \si{\joule}
\choice \si{\watt}
\choice cal
\choice kcal
\choice BTU
\end{choices}

\question[5] Two 6S racing quadrotors (mass \SI{700}{\gram}) are heading away each other on reciporical headings at 165 mph (\SI{74}{\meter\per\second}). What is the total momentum of the system? 
\begin{choices}
\CorrectChoice Zero, since they are heading straight away from each other
\choice \SI{104}{\kilo\gram\meter\per\second}
\choice \SI{10400}{\kilo\gram\meter\per\second}
\choice Zero, because momentum is conserved
\end{choices}

\question[5] Which of the following is NOT a situation where you might apply momentum conservation. 
\begin{choices}
\choice Two sumo wrestlers at the instant of collision
\choice A plunge-diving gannet (family \emph{Sulidae}) at the instant of colliding with the water
\choice A neutron colliding with a hydrogen nucleus
\choice Modeling the thrust produced by a rocket
\CorrectChoice When determining the position of a ball, given its velocity, acceleration, and initial position. 
\end{choices}

\question[5] When Marcello punches a punching bag, he pivots his hip and pushes off with the ball of his foot. Why?
\begin{choices}
\CorrectChoice By maximizing his momentum ($m\Delta v$) he increases the force he is able to generate on contact with the bag ($F\Delta t$). 
\choice Because it looks cute in a TikTok. 
\choice To minimize the recoil on himself, by making his overall momentum close to zero. 
\choice There is no functional role of hip or foot motion. 
\choice When he extends his arm, he must move his hips in order to remain balanced with his center of gravity firmly within the base provided by his stance. 
\end{choices}

\clearpage
\question[5] Assuming all else equal, what weapon should have the greatest recoil?  
\begin{choices}
\choice A 0.50 caliber sniper rifle (660 grains, \SI{43}{\gram}). 
\choice M-14 firing a 7.61x51 NATO round (140 grains, \SI{10}{\gram})
\choice M-16A2 firing a 5.56x45 NATO round (62 grains, \SI{4}{\gram})
\choice M9 firing a 9x19 Parabellum round (115 grains, \SI{7.45}{\gram})
\CorrectChoice A BB-62 Class battleship firing a 16-inch run (\SI{1200}{\kilo\gram})
\end{choices}

\question[5] Two male \emph{\dag Pachycephalosaurus} dinosaurs are hypothesized to have head butted each other in battles to secure mates.  Imagine one dinosaur had a lower mass and a lower speed heading into the bout. If you were a female \emph{\dag Pachycephalosaurus} and modeled the collision as an inelastic collision, which challenger would you pick?
\begin{choices}
\choice The ``give way'' male, because he is more sensitive. 
\CorrectChoice The ``stand on'' male, because it is an honest signal that he has higher mass and higher speed and is thus a higher quality mate.
\choice The ``stand on'' male, because he is funny and makes you laugh. 
\choice The ``give way'' male, because it is an honest signal that he has higher mass and higher speed and is thus a higher quality mate.
\choice The ``give way'' male, because altruism is good for improving the species. 
\end{choices}

\question[5] Xuan the puppy walks \SI{1}{\meter} in \SI{1}{\second}. What is his acceleration? 
\begin{choices}
\choice 1
\choice \SI{1}{\meter\per\second}
\choice \SI{1}{\meter\per\second\squared}
\choice \SI{1}{\meter} east
\CorrectChoice do not have enough information
\end{choices}

\question[5]  A marble is launched at a speed $V$ and angle $\theta$ to the horizontal. What is the horizontal component of its velocity? 
\begin{choices}
\choice $V$
\choice $0$
\CorrectChoice $V\cos(\theta)$
\choice $V\sin(\theta)$
\choice $V\tan(\theta)$
\end{choices}




\clearpage
\question While playing spikeball, a student finds that a ball dropped onto the net from \SI{2}{\meter} high bounces back to \SI{1.4}{\meter}
\begin{parts}
\part[5] What is the initial potential energy of the ball? 
\part[5] What is the kinetic energy of the ball just before it hits the net and bounces back. 
\part[5] What is the final potential energy of the ball?
\part[5] How much energy was lost during the rebound off the net? 
\end{parts}




\clearpage
\question For the second short answer question, you get a job at NASA working on a small, zero G robot propelled with compressed air, able to move around inside the International Space Station. You are planning a set of commands to transmit to the robot for an upcoming flight test on a ``Vomit Comet'' C-9B microgravity simulator plane. 

\begin{parts}
\part[10] The first test maneuver is to fly up \SI{10}{\meter} in \SI{10}{\second}, hover for 
\SI{10}{\second}, then return to the launch point in \SI{1}{\second}. In the space here, sketch the vertical position and velocity of the robot. (You will need this to construct velocity commands to send to the robot, and to set failsafe limits on position in its autonomous flight controller.) 
\begin{solution}[3in]
\end{solution}

\part[10] During the actual test, a failure occurs in the propulsion system, resulting in the throttle valve being stuck open during the first part of the maneuver (heading upwards). With the failed valve, the force on the robot is \SI{1}{\newton} and its mass is \SI{1}{\kilo\gram}. Please plot the robot's position versus time. The far wall of the test chamber is \SI{20}{\meter} away; show when it hits the wall on your graph. Assume the robot is streamlined so that air resistance is small and neglect any change in mass from the expulsion of propellant.  
\begin{solution}
\end{solution}
\end{parts}


\clearpage
\question
\begin{parts}
\part[10] In your own words, describe what is the difference between momentum and energy? 
\begin{solution}[4in]
\end{solution}
\part[10] In your own words, describe what is the difference between position, velocity, and acceleration? 
\begin{solution}[4in]
\end{solution}
\end{parts}
\end{questions}
\end{document}
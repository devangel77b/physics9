\documentclass[hw,addpoints,noanswers]{exam}

\usepackage{../physics9}

\title{(daily) Homework: Collision analysis}
\author{\mobeardInstructorShort}
\date{\today}
\duedate{\printdate{4/13/2021}}

\renewcommand{\arraystretch}{1.5}

\begin{document}
\maketitle

Read from lesson 2 of the Momentum and Collisions chapter at the Physics Classroom, \url{http://www.physicsclassroom.com/Class/momentum/u4l2d.html} and \url{http://www.physicsclassroom.com/Class/momentum/u4l2e.html}. These questions are adapted from The Physics Classroom. 

\textbf{Please show your work for all questions.}

\begin{questions}
\question
A \SI{10}{\kilo\gram} medicine ball is thrown at a velocity of \SI{15}{\kilo\meter\per\hour} to a \SI{50}{\kilo\gram} skater who is at rest on ice. The skater catches the ball and subsequently slides with the ball across the ice. Consider the skater and the ball as two separate parts of an isolated system. (no external forces). 
\begin{parts}
\part[1] Fill in the before- and after-collision table below.
\begin{table}[h]
\ifprintanswers
First, convert \SI{15}{\kilo\meter\per\hour} to \SI{4.2}{\meter\per\second}. Then compute $p=mv$ before the collision and apply \textbf{momentum conservation} to get total $p$ after the collision. The change in total momentum will be 0. 
\begin{center}
\begin{tabular}{|l|p{1in}|p{1in}|p{1in}|} \hline
& $p$ before collision & $p$ after collision & $\Delta p$ \\ \hline
ball & 42 & & \\ \hline
skater & 0 & & \\ \hline
total &42 & 42 & 0 \\ \hline
\end{tabular}
\end{center}
Now compute the final velocity of the ball+skater since $(10+50)v = 42$, so $v=\SI{0.7}{\meter\per\second}$ for an \textbf{inelastic collision}. We can now fill in the rest of the table, using $p=mv$ to find the momentum of the ball and skater after the collision. 
\begin{center}
\begin{tabular}{|l|p{1in}|p{1in}|p{1in}|} \hline
& $p$ before collision & $p$ after collision & $\Delta p$ \\ \hline
ball & 42 & \textbf{7} & \textbf{-35} \\ \hline
skater & 0 & \textbf{35} & \textbf{+35} \\ \hline
total & 42 & 42 & 0 \\ \hline
\end{tabular}
\end{center}
\else
\begin{center}
\begin{tabular}{|l|p{1in}|p{1in}|p{1in}|} \hline
& $p$ before collision & $p$ after collision & $\Delta p$ \\ \hline
ball & & & \\ \hline
skater & & & \\ \hline
total & & & \\ \hline
\end{tabular}
\end{center}
\fi
\end{table}
\part[1] Determine the velocity of medicine ball and the skater after the collision. 
\begin{solution}[1in]
\begin{align*}
\SI{42}{\newton\second} &= (\SI{10}{\kilo\gram}+\SI{50}{\kilo\gram}) v_f \\
v_f &= \SI{0.7}{\meter\per\second}.
\end{align*}

The solution for the other questions follows the exact same steps as this one. 
\end{solution}
\end{parts}

\ifprintanswers\clearpage\fi
\question
A large fish with a mass of \SI{1}{\kilo\gram} is in motion at \SI{45}{\centi\meter\per\second} when it encounters a smaller fish ($m=\SI{0.25}{\kilo\gram}$) that is at rest. The large fish swallows the smaller fish and continues in motion at a reduced speed.
\begin{parts}
\part[\half] Fill in the before- and after-collision table below.
\begin{table}[h]
\ifprintanswers
\begin{center}
\begin{tabular}{|l|p{1in}|p{1in}|p{1in}|} \hline
& $p$ before collision & $p$ after collision & $\Delta p$ \\ \hline
large fish & 0.45 & 0.36 & -0.09 \\ \hline
small fish & 0 & 0.09 & +0.09 \\ \hline
total & 0.45 & 0.45 & 0 \\ \hline
\end{tabular}
\end{center}
\else
\begin{center}
\begin{tabular}{|l|p{1in}|p{1in}|p{1in}|} \hline
& $p$ before collision & $p$ after collision & $\Delta p$ \\ \hline
large fish & & & \\ \hline
small fish & & & \\ \hline
total & & & \\ \hline
\end{tabular}
\end{center}
\fi
\end{table}
\part[\half] Determine the velocity of the large and the small fish after the collision.
\begin{solution}[1in]
\begin{align*}
\SI{0.45}{\kilo\gram\meter\per\second} &= (\SI{1}{\kilo\gram}+\SI{0.25}{\kilo\gram}) v_f \\
v_f &= \SI{0.36}{\meter\per\second}.
\end{align*}
\end{solution}
\end{parts}

\clearpage
\section*{Momentum and collisions}
\question
A \SI{0.150}{\kilo\gram} baseball moving at a speed of \SI{45.0}{\meter\per\second} crosses the plate and strikes the \SI{0.250}{\kilo\gram} catcher's mitt (originally at rest). The catcher's mitt immediately recoils backwards (at the same speed as the ball) before the catcher applies an external force to stop its momentum. If the catcher's hand is in a relaxed state at the time of the collision, it can be assumed that no net external force exists and the law of momentum conservation applies to the baseball-catcher's mitt collision. 
\begin{parts}
\part[\half] Fill in the before- and after- collision table below.
\begin{table}[h]
\ifprintanswers
\begin{center}
\begin{tabular}{|l|p{1in}|p{1in}|p{1in}|} \hline
& $p$ before collision & $p$ after collision & $\Delta p$ \\ \hline
baseball & 6.75 & 2.53 & -4.22 \\ \hline
catcher's mitt & 0 & 4.22 & +4.22 \\ \hline
total & 6.75 & 6.75 & 0 \\ \hline
\end{tabular}
\end{center}
\else
\begin{center}
\begin{tabular}{|l|p{1in}|p{1in}|p{1in}|} \hline
& $p$ before collision & $p$ after collision & $\Delta p$ \\ \hline
baseball & & & \\ \hline
catcher's mitt & & & \\ \hline
total & & & \\ \hline
\end{tabular}
\end{center}
\fi
\end{table}
\part[\half] Determine the velocity of the baseball/catcher's mitt immediately after the collision.
\begin{solution}[1in]
\begin{align*}
\SI{6.75}{\kilo\gram\meter\per\second} &= (\SI{0.15}{\kilo\gram}+\SI{0.25}{\kilo\gram}) v_f \\
v_f &= \SI{16.88}{\meter\per\second}.
\end{align*}
\end{solution}
\end{parts}

\question
 A \SI{4800}{\kilo\gram} truck traveling with a velocity of \SI{+4.0}{\meter\per\second} collides head-on with a \SI{1200}{\kilo\gram} car traveling with a velocity of \SI{-12}{\meter\per\second}. The truck and car entangle and move together after the collision. 
 \begin{parts}
 \part[\half] Fill in the before- and after- collision table below.
 \begin{table}[h]
\begin{center}
\begin{tabular}{|l|p{1in}|p{1in}|p{1in}|} \hline
& $p$ before collision & $p$ after collision & $\Delta p$ \\ \hline
truck & 19200 & 3840 & -15360 \\ \hline
car & -14400 & 960 & +15360 \\ \hline
total & 4800 & 4800 & 0 \\ \hline
\end{tabular}
\end{center}
\end{table}
\part[\half] Determine the velocity of the truck and car immediately after the collision.
\begin{solution}[1in]
\begin{align*}
\SI{4800}{\kilo\gram\meter\per\second} &= (\SI{4800}{\kilo\gram}+\SI{1200}{\kilo\gram}) v_f \\
v_f &= \SI{0.8}{\meter\per\second}.
\end{align*}
Be careful with the signs to reflect that the car and the truck are traveling in different directions. 
\end{solution}
\end{parts}
\end{questions}
\end{document}
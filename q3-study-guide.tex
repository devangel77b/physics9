\documentclass[handout]{tufte-handout}

\usepackage[handout]{physics9}

\title{Q3 study guide: mechanics/kinematics}
\date{\today}
\author{\mobeardInstructorShort}

\begin{document}
\maketitle
\section{Kinematics}
Kinematics is the quantitative study of motion. To describe how an object moves, we typically describe its position, velocity, and acceleration in some useful coordinate system / coordinate frame. 
\subsection{Position, velocity, acceleration}
\textbf{Position} is a vector and describes where an object is in space relative to an established coordinate system. Typical SI units for position are \si{\meter}. I normally use $\vec{r}$, $\vec{x}$, or $\vec{y}$ as variables to describe position, often with an arrow over them to remind me they are vectors. 

\textbf{Velocity} is also a vector and describes the \textbf{time rate of change of position}. Its units are \si{\meter\per\second}. I normally use $\vec{v}$ to represent velocity. Considering $\Delta$ or $d$ as a ``change in'', velocity becomes
\begin{align}
\text{velocity} &= \dfrac{\text{change in position}}{\text{change in time}} \\
&= \dfrac{\Delta \vec{x}}{\Delta t} \\
&= \dfrac{d\vec{x}}{dt}
\end{align}
The last form, read ``dee x dee tee,'' is how velocity is typically written as a ``derivative'' in \textbf{calculus}, an advanced type of math that was invented partially to make physics easier to understand. The ``time rate of change of position'' relationship means that \textbf{velocity is like the slope of a position versus time graph,} and that \textbf{position is like the area under a velocity versus time graph.} 

\textbf{Acceleration} is obtained from doing a Madlibs sort of thing... we take it as the \textbf{time rate of change of velocity}. Its units are \si{\meter\per\second\squared}. I normally use $\vec{a}$ to represent acceleration. 
\begin{align}
\text{acceleration} &= \dfrac{\text{change in velociy}}{\text{change in time}} \\
&= \dfrac{\Delta \vec{v}}{\Delta t} \\
&= \dfrac{d \vec{v}}{dt}
\end{align}
The ``time rate of change of velocity'' relationship means that \textbf{acceleration is like the slope of a velocity versus time graph,} and that \textbf{velocity is like the area under an acceleration versus time graph.} 

\subsection{Related scalar quantities}
Vector position or \textbf{displacement} are related to the scalar quantity \textbf{distance}.  \textbf{Distance} will normally be considered to be the total distance traveled along a path from $\vec{x}_0$ to $\vec{x}_f$ and is only a simple number (e.g. 50 miles) so it is a scalar quantity.  \textbf{Displacement} is a vector quantity and is normally taken as $\vec{x}_f - \vec{x}_0$. 

Vector velocity is related to the scalar quantity \textbf{speed}. Speed is normally either the scalar magnitude of velocity, e.g. instantaneous speed, $|v|$, or the total distance traveled divided by the total time (average speed). 

\subsection{1D motion with constant velocity}
The following equations hold for 1D motion at \textbf{constant velocity}, which means the speed \emph{and direction} of the object are not changing:
\begin{align}
x(t) &= v t + x_0 \\
v(t) &= v\ \text{(constant)} \\
a(t) &= 0
\end{align}
Examples of 1D motion at constant velocity would include things like a skier moving at \SI{5}{\meter\per\second} north; a softball in space with no forces acting on it; or an object that is not accelerating. The big example fo this is when we pushed people on chairs at constant speed, and also the horizontal component of the marble shooting experiment. 

\subsection{1D motion with constant acceleration}
The following equations hold for 1D motion with \textbf{constant (linear) acceleration}, which means there is a net force acting on the object that makes it go faster or slower. 
\begin{align}
x(t) &= \dfrac{1}{2} a t^2 + v_0 t + x_0 \\
v(t) &= at + v_0 \\
a(t) &= a\ \text{(constant)} 
\end{align}
Examples of 1D motion with constant acceleration include the case of a ball dropped from the second floor balcony, or a car at a stop light when it hits the accelerator and before shifting gears, or a rocket ship firing a thruster with a specified force output. The big example of this is when we dropped stuff from the balcony; as well as the vertical component of the marble shooting experiment. 

\textbf{For this test, do not expect 2D or 3D motion; or motions that cannot be modeled as either constant velocity or constant acceleration.}

\section{Forces}
\textbf{Forces} can come from things like weight (gravity), aerodynamic lift or drag, thrust from an engine, friction from the ground, normal forces from the ground; forces also arise from charges, electric and magnetic fields, reaction forces, etc etc. The SI unit of force is a \si{\newton}, defined as \SI{1}{\kilo\gram\meter\per\second\squared}. You may also see force specified as pounds force (lbf) when working with non-SI units (such as in specifying the thrust of a jet engine). The form of Newton's second law used in Physics 9 also gives 
\begin{align}
\text{net force} &= \text{mass} \cdot \text{acceleration} \\
\sum\vec{F} &= m\vec{a}
\end{align}

\subsection{Free body diagrams}
\textbf{Free body diagrams} are ways to visualize the force acting on an object. The object is drawn isolated from the rest of the universe, and arrows are used to show the forces that are acting on it and their location and directions. This is a useful tool in analyzing the mechanics of all sorts of things, but \textbf{for this test you may only be asked to draw free body diagrams of very simple situations. }

\section{Newton's laws}
\begin{enumerate}
\item An object at rest will stay at rest, unless acted upon by an outside force; an object in motion will stay in motion unless acted upon by an outside force. 
\item If there is an outside force acting, the sum of the forces will equal the time rate of change of momentum\sidenote{Momentum is $\vec{p} = m \vec{v}$ and is the product of mass and velocity.}. Since in Physics 9 we are usually dealing with objects of constant mass, a simpler version of this law is just
\begin{equation}
\sum\vec{F} = m \vec{a} 
\end{equation}
where $\sum\vec{F}$ is the net force or the sum of the forces, $m$ is mass, and $\vec{a}$ is acceleration. 
\item For every action, there is an equal and opposite reaction. 
\end{enumerate}

The first law could also be viewed as the case of the second law where $\vec{a}=0$, and (together with the third law) is studied most in ``statics''\sidenote{First course taken by many engineers in college...}, when considering the balance of forces for and within objects that are not accelerating. 

The second law shows up most in cases where objects are accelerating, such as in studies of vehicles, maneuvers, propulsion systems, or generally in ``dynamics''\sidenote{Second course taken by many engineers in college...}. 

\end{document}
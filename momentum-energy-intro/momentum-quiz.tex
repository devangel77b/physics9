\documentclass[quiz,addpoints,noanswers]{exam}

\usepackage[quiz]{../physics9}

\title{Quiz: Momentum}
\date{\today}
\author{\mobeardInstructorShort}

\begin{document}
\maketitle
\begin{abstract}
To help with your problem solving skills and build confidence in applying physics to actual situations we will start having more quizzes; this will also help even out the grading. The quizzes are meant to be low stress and short; you may use your notes to help recall concepts you may have missed but try your best to answer without lookups. You should not need to spend more than \SI{5}{\min} on this quiz. 
\end{abstract}

\begin{questions}
\question Xuan the Guide Dog Puppy (mass \SI{65}{\pound}, \SI{29.4}{\kilo\gram}) is in the park when he is given the recall command \textbf{come!}. He begins running back towards his handler at \SI{8}{\meter\per\second}. 
\begin{parts}
\part[1] What is the formula for momentum? \answerline[$\vec{p}=m\vec{v}$]
\part[1] What is Xuan's mass $m$? \SI{1}{\kilo\gram} is \SI{2.2}{\pound}. \answerline[$m=\SI{29.4}{\kilo\gram}$]
\part[1] What is Xuan's momentum? Please show your work.
\begin{solution}[2in]
\begin{align}
\vec{p} &= m \vec{v} \\
&= \SI{29.4}{\kilo\gram} \cdot \SI{8}{\meter\per\second}\ \text{towards his handler} \\
&= \SI{235}{\kilo\gram\meter\per\second}
\end{align}
\end{solution}
%\part Is momentum a scalar or a vector? \answerline[vector]

Binzel the Therapy Dog (mass \SI{80}{\pound}, \SI{36.3}{\kilo\gram}) is sitting patiently next to his handler when Xuan collides with him. Assume momentum is \textbf{conserved} during the collision, i.e. the total momentum of the system of Binzel and Xuan together is constant. 
\part[1] What was Binzel's initial velocity? What was Binzel's initial momentum? What was the momentum of the system before the collision? What was the momentum of the system after the collision? Please show your work. 
\begin{solution}[2in]
Binzel's initial velocity is \SI{0}{\meter\per\second}; his initial momentum is therefore \SI{0}{\kilo\gram\meter\per\second}. The momentum of the system before and after the collision is therefore \SI{235}{\kilo\gram\meter\per\second}. 
\end{solution}
\end{parts}
\end{questions}
\end{document}
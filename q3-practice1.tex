\documentclass[exam,addpoints]{exam}

\usepackage[exam]{physics9}

\title{Practice Test \#3A}
\date{\today}
\author{\mobeardInstructorShort}

\begin{document}
\maketitle
\vfill
\mobeardExamNameBlock
\vfill
Instructions: 
\begin{enumerate}
\item Do not open exam until instructor announces that you may begin.
\item Closed notes, closed book.  You may use a two \SI{8.5x11}{\inch} note sheets, both sides.
\item Please write your name on each page in case the sheets are separated. 
\end{enumerate}
\vfill
\begin{center}
\gradetable[h][questions]
\end{center}
\clearpage

\begin{questions}
\question[1] Order these categories from most specific to most general:
\begin{choices}
\choice species
\choice domain
\choice genus
\choice kingdom
\choice family
\choice phylum
\choice order
\choice class
\end{choices}

\question[1] Refer to the phylogenetic tree. Which statement is consistent with the tree?
\begin{choices}
\choice Humans evolved from birds.
\choice Humans evolved from lizards.
\choice Humans evolved from chimps.
\CorrectChoice Humans and chimps share a common ancestor. 
\end{choices}

\question[1] Which of the following statements regarding evolution is \textbf{FALSE}?
\begin{choices}
\CorrectChoice Evolution has a ``drive'' and always acts to improve performance.
\choice Evolution is descent with modification.
\choice Evolution is a consequence of having heritable traits and differential survival of offspring.
\choice Evolution occurs over geologic timescales, but can also occur rapidly such as in COVID-19.
\end{choices}

\question[1] Based on \emph{scientific evidence}, approximately how old is life on earth?
\begin{choices}
\CorrectChoice Prokaryotic life is around 3.9 billion years old, multicellular organisms have only been around for about 500 million years.
\choice 5000 years.
\choice 17 years.
\choice Everything is 1 day old and was created yesterday by an evil genius out to trick us. 
\end{choices}

\question[1] Which molecule is best described as the energy ``currency'' used within cells?
\begin{choices}
\CorrectChoice Adenosine triphosphate.
\choice Adenosine diphosphate.
\choice Glucose.
\choice Nicotinamide adenine dinucleotide + hydrogen. 
\end{choices}

\clearpage
\question[1] Which of the following is \textbf{NOT} a way that single cells can bend the limits of diffusion?
\begin{choices}
\choice Being large in one dimension but skinny in others.
\choice Using ATP to power transmembrane proteins to actively pump materials into a cell.
\choice Using special organelles or increasing flow to improve intake / output of materials. 
\CorrectChoice By growing very large, cells can avoid the need to use diffusion. 
\end{choices}

\question[1] Which of the following is most consistent with the flow of energy in an ecosystem?
\begin{choices}
\choice Because cellular energetic processes are 100\% efficient, we expect to see equal biomass of primary producers, grazers, and predators.
\CorrectChoice In terrestrial ecosystems, we expect to see the highest biomass of primary producers, next grazers, and the least biomass will be predators.
\choice The particular cellular energetic processes used at the individual cell level have no bearing on the abundance and distribution of life at the ecosystem scale. 
\choice While some small percentage of energy comes from the sun, a more significant fraction comes from fossil fuels like methane exuded at hydrothermal vents. 
\end{choices}

\question[1] Which of the following about photosynthesis is \textbf{FALSE}?
\begin{choices}
\choice Light reactions do not require light to proceed, they are named for Calvin Light, FRS, who was awarded the Nobel Prize in 1953 for his work on photosynthesis.
\choice CAM and C4 processes are alternatives to the Calvin cycle that allow CO2 uptake in the dark, such as at night. 
\choice Overall, photosynthesis uses energy from photons to synthesize CO2 and water into glucose, liberating oxygen. 
\choice In eukaryotic photoautotrophs, the chloroplast is the organelle which is the site of photosynthesis and is believed to have arisen in an endosymbiotic event with a blue-green alga. 
\end{choices}

\question[1] Which is likely to be most severe, and why?
\begin{choices}
\choice Drowning in seawater, because isotonic solutions can cause cells to burst
\choice Drowning in fresh water, because hypotonic solutions can cause cells to burst
\choice Drowning in fresh water, because hypotonic solutions can cause cells to shrink or dry out
\choice Drowning in fresh water, because  hypertonic solutions can cause cells to shrink or dry out
\end{choices}

\question[1] What are three processes for getting materials into or out of a cell?
\begin{choices}
\CorrectChoice simple diffusion, facilitated diffusion, active transport
\choice ingestion, digestion, excretion
\choice injection, ejection, abjection
\choice transport, diffusion, advection
\end{choices}





\clearpage
\question[23] In the space provided, explain what you know about photosynthesis. Pictures encouraged. Relate your explanation to something about how an organism lives.
\begin{solution}[6in]
\end{solution}

\clearpage
\question[23] In the space provided, explain what you know about cellular respiration. Pictures encouraged. Relate your explanation to something about how an organism lives. 
\begin{solution}[6in]
\end{solution}

\clearpage
\question[22] When making gravlax, a piece of salmon filet (the main swim muscle of a salmon) was placed in a bag with salt and sugar. Over time, the piece lost mass and exuded water. Using what you know about osmosis and diffusion, please explain what happened and why. Speculate: how does this process preserve fish from spoilage? 
\begin{solution}[6in]
\end{solution}

\clearpage
\question[22] We discussed photosynthesis (which uses light energy and produces glucose and oxygen) and cellular respiration (which consumes sugars).  Can life survive without oxygen? Support your hypothesis with examples from the history of life on earth, biochemistry examples, or example organisms. 
\begin{solution}[6in]
\end{solution}

\end{questions}
\end{document}
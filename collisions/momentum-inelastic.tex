\documentclass[handout]{tufte-handout}

\newcommand{\myroot}{..}
\usepackage[handout]{\myroot/physics9}
\usepackage{textcomp}

\title{Momentum -- how to solve inelastic collision problems}
\date{\today}
\author{\mobeardInstructorShort}

\begin{document}
\maketitle

\newthought{Momentum conservation} gives a nice way to understand \textbf{collisions}. When two bodies collide, the total momentum of the two bodies before the collision is equal to the total momentum afterwards. 

\section{Types of collisions}
Today we will deal with \textbf{inelastic collisions}\sidenote{Inelastic collisions are not the only possibility; in an \textbf{elastic collision}, the bodies bounce off each other and do not stick together. Another possibility is an \textbf{explosion}, when a single body breaks into two or more that fly off in different directions.}, which are when the two bodies stick together after colliding. 

\subsection{Momentum is conserved in an inelastic collision}
\textbf{Momentum is conserved in an inelastic collision.} Let us consider the case of two masses, $m_1$ and $m_2$, with velocities $v_1$ and $v_2$, respectively\sidenote{For now we will only consider 1D motion.}. The initial momentum is given by 
\begin{equation}
p_0 = m_1 v_1 + m_2 v_2\ \text{initial momentum}.
\end{equation} 

The two masses stick together, forming a single mass equal to $m_1+m_2$. The final velocity is then obtained from momentum conservation
\begin{equation}
p_f = (m_1 + m_2) v_f = p_0.
\end{equation}
 
 We can substitute for $p_0$ and solve for $v_f$:
 \begin{align}
 (m_1 + m_2) v_f &= m_1 v_1 + m_2 v_2 \\
 \dfrac{1}{\cancel{m_1 + m_2}} \cancel{(m_1 + m_2)} v_f &= \dfrac{1}{(m_1 + m_2)} (m_1 v_1 + m_2 v_2) \\
 v_f &= \dfrac{m_1 v_1 + m_2 v_2}{m_1 + m_2} .
 \end{align}

\subsection{Energy decreases in an inelastic collision}
\textbf{Energy decreases during an inelastic collision.} The kinetic energy before the collision is 
\begin{equation}
E_0 = \frac{1}{2} m_1 v_1^2 + \frac{1}{2} m_2 v_2^2
\end{equation}

Some energy is lost when the two bodies stick together. The kinetic energy after the collision is\sidenote{If you were to look a frame moving at the average speed of the two bodies, you would see the kinetic energy in this frame must be eaten up when the two bodies stick.}
\begin{align}
E_f &= \frac{1}{2} (m_1 + m_2) v_f^2 \\
&= \frac{1}{2} \dfrac{(m_1 v_1 + m_2 v_2)^2}{m_1 + m_2} 
\end{align}

\end{document}
\documentclass [hw,addpoints,noanswers]{exam}

\usepackage[hw]{../physics9}


\title{Lab report: Is momentum conserved? }
\author{\mobeardInstructorShort}
\date{\printdate{4/29/2021}}
\duedate{\printdate{5/12/2021}}

\sisetup{separate-uncertainty=true,per-mode=symbol}


\begin{document}
\maketitle

\begin{abstract}
Now that you have created figures using the data from the momentum experiment, you will write a lab report. In the lab report, you will address the question, ``Is momentum conserved?'' To help you, we have written most of the lab report, but you are responsible for (1) the final sentence of the abstract; (2) a paragraph in the discussion section addressing if momentum is conserved or not; and (3) a paragraph in the discussion section that deals with sources of experimental error.  For this assignment, as the first author and corresponding author you are in a position of power; you are now the world expert in answering the question ``Is momentum conserved?'' The world awaits your discussion of the significance of your findings. 
\end{abstract}

\section{Introduction}
You have designed an experiment and collected data; but any knowledge gained is useless unless it is communicated with the rest of the world. This is typically done in writing, in the form of a lab report, or, more typically, a peer-reviewed article published in a scientific journal. Now is your chance to try your hand at writing one of these.  There are few sections in a typical scientific journal article: 
\begin{enumerate}
\item Title, author list, and other bibliographic information. 
\item \textbf{Abstract}, usually a single paragraph, that summarizes the work and the findings. It is often the first thing read by others, who decide whether or not to read the rest of your article based on the abstract. 
\item \textbf{Introduction}, which sets up the research question to be answered and the context or reasons it is of interest. 
\item \textbf{Methods and materials}, which explain how the experiment was done in enough detail that it can be \textbf{replicated}. This is a key principle in science in general (not just physics).
\item \textbf{Results}, which present the results of what was observed. 
\item \textbf{Discussion} of the significance of the findings or conclusions reached from the results is separate from the results. 
\item Articles usually include acknowledgements as well as any citations. 
\item A major part of any article are the \textbf{figures and tables} that communicate the results. You have already worked on these as part of the previous assignment; a friendly scientist has updated and helped to format your figures based on what you learned. 
\end{enumerate}

\clearpage
\section{Deliverables}
\textbf{You have a week} to write your assigned sections; arrange now if you need help from the Center for Academic Writing or other assistance.  
\begin{questions}
\question \textbf{Read} the draft manuscript ``Is momentum conserved?" and the examples provided by your instructor. See if you can see the structure of how they are organized. 
\question Write the \textbf{last sentence of the abstract}. 
\question Write the \textbf{first paragraph of the discussion}, dealing with ``Is momentum conserved?" 
\question Write the \textbf{second paragraph of the discussion}, dealing with ``Sources of experimental error''. 
\end{questions}
When you have completed your sections, you will submit them online through Google Classroom. Your manuscript will then be provided to two of your peers for peer review as the next assignment; thus \textbf{late submissions will not be accepted!}

\section{Some technical writing advice}
Often, students hit a block and are unsure what to write. Since you will be revising this through a peer review process, try not to worry about making it sound perfect. Get the thoughts down first, then we can help you polish your thinking.

Your audience is generally smart, science-interested people. They may not have done the exact experiment you are discussing so you need to describe details well enough for them to follow; but otherwise assume they have similar technical knowledge to you. 

Sometimes, it is also helpful to think you you might explain what you did to (Lolo and Lola, grandma, dad, Aunt Ruthie, the dog, etc). If you can explain it to them then you can explain it to others. 

Write simply. There is no need to use gargantuan SAT words or to obfuscate your writing to sound inflatedly erudite. Use active voice. First person is actually OK too. Just be clear. 

Read a lot. You'll gradually pick up your own voice and style, building on what you read and your own personality. 
\end{document}
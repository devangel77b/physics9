\documentclass [hw]{exam}

\usepackage[hw]{physics9}

\title{Practice problems: Kinematics in one dimension}
\author{\mobeardInstructorShort}
\date{\today}
\duedate{never}

\begin{document}
\maketitle

\begin{questions}
\question A car's odometer reads \SI{22687}{\kilo\meter} at the start of a trip and \SI{22791}{\kilo\meter} at the end. The trip took \SI{4}{\hour}. What was the car's average speed in \si{\kilo\meter\per\hour}? In \si{\meter\per\second}?

\question An auto travels at \SI{25}{\kilo\meter\per\hour} for \SI{4}{\minute}, then at \SI{50}{\kilo\meter\per\hour} for \SI{8}{\minute}, and finally at \SI{20}{\kilo\meter\per\hour} for \SI{2}{\minute}. Find the total distance covered in \si{\kilo\meter} and the average speed for the complete trip in \si{\meter\per\second}. 

\question Use dimensional analysis to determine which of the following equations is certainly wrong:
\begin{equation*}
\lambda=vt\quad F=\frac{m}{a}\quad F=\frac{mv}{t}\quad h=\frac{v^2}{2g}\quad v=(2gh)^{1/2}
\end{equation*} 
where $\lambda$ and $h$ are lengths, and $[F]=[MLT^{-2}]$. The other symbols have their usual meanings.

\question A body with initial velocity \SI{8}{\meter\per\second} moves along a straight line with constant acceleration and travels \SI{640}{\meter} in \SI{40}{\second}. For the \SI{40}{\second} interval, find the average velocity, the final velocity, and the acceleration. 

\question A truck starts from rest and moves with a constant acceleration of \SI{5}{\meter\per\second\squared}. Find its speed and the distance traveled after \SI{4}{\second} have elapsed. 

\question A box slides down an incline with uniform acceleration. It starts from rest and attains a speed of \SI{2.7}{\meter\per\second} in \SI{3}{\second}. Find the acceleration and the distance moved in the first \SI{6}{\second}.

\question A car starts from rest and coasts down a hill with constant acceleration. If it goes \SI{90}{\meter} in \SI{8}{\second}, find the acceleration and the velocity after \SI{1}{\second}.

\question A plane starts from rest and accelerates along the ground before takeoff. It moves \SI{600}{\meter} in \SI{12}{\second}. Find the acceleration, the speed at the end of \SI{12}{\second}, and the distance moved during the twelfth second. 

\question A train running at \SI{30}{\meter\per\second} is slowed uniformly to a stop in \SI{44}{\second}. Find the acceleration and the stopping distance.

\question A rocket-propelled car starts from rest at $x=0$ and moves in the $+\hat{x}$ direction with constant acceleration $a=\SI{5}{\meter\per\second\squared}$ for \SI{8}{\second} until the fuel is exhausted. It then continues to move with constant velocity. What distance does the car cover in \SI{12}{\second}?

\question A body falls freely from rest. Find its acceleration, the distance it falls in \SI{3}{\second}, its speed after falling \SI{70}{\meter}, the time required to reach a speed of \SI{25}{\meter\per\second}, and the time taken to fall \SI{300}{\meter}/ 

\question A ball dropped from a bridge strikes the water in \SI{5}{\second}. Find the speed with which it strikes and the height of the bridge. 

\question The acceleration due to gravity on the moon is \SI{1.67}{\meter\per\second\squared}. If a person can throw a stone \SI{12.0}{\meter} straight upward on the earth, how high should the person be able to throw the stone on the moon? Assume that the throwing speeds are the same in the two cases. 
\end{questions}
\end{document}
\documentclass[avery5371,grid]{flashcards}
% avery5371 or avery5388
% dedicated to Flores, Mejia, Dinofrio, and Lee

\usepackage{physics9}

\begin{document}
% complex numbers
%\begin{flashcard}{$s_1=a+bj$, $s_2=c+dj$, find sum $s_1+s_2$}
%$s_1+s_2=(a+c)+(b+d)j$
%\end{flashcard}

\begin{flashcard}{position}
$\vec{x}$
\end{flashcard}
\begin{flashcard}{$\vec{x}$ units}
\si{\meter}
\end{flashcard}

\begin{flashcard}{velocity}
$\vec{v} = \dfrac{\Delta \vec{x}}{\Delta t}$
\end{flashcard}
\begin{flashcard}{$\vec{v}$ units}
\si{\meter\per\second}
\end{flashcard}

\begin{flashcard}{acceleration}
$\vec{a} = \dfrac{\Delta \vec{v}}{\Delta t}$
\end{flashcard}
\begin{flashcard}{$\vec{a}$ units}
\si{\meter\per\second\squared}
\end{flashcard}

\begin{flashcard}{force}
$\vec{F}=m\vec{a}$ (Newton's 2nd law)
\end{flashcard}
\begin{flashcard}{$\vec{F}$ units}
\si{\newton}
\end{flashcard}
\begin{flashcard}{mass}
$m$
\end{flashcard}
\begin{flashcard}{$m$ units}
\si{\kilo\gram}
\end{flashcard}
\begin{flashcard}{What is a freebody diagram?}
A diagram showing the forces (magnitude and direction) acting on an isolated body
\end{flashcard}
\begin{flashcard}{weight}
$W=mg$ or force due to gravity acceleration
\end{flashcard}
\begin{flashcard}{weight units}
\si{\newton}
\end{flashcard}
\begin{flashcard}{acceleration of gravity at Earth's surface}
$g = \SI{-9.81}{\meter\per\second\squared}$
\end{flashcard}

\begin{flashcard}{Newton's first law}
An object at rest will stay at rest, and an object in motion will stay in motion, unless acted upon by an outside force
\end{flashcard}
\begin{flashcard}{Newton's second law (long version)}
The sum of the forces acting on a body is equal to the time rate of change of its momentum
\end{flashcard}
\begin{flashcard}{Newton's second law (Physics 9 version)}
$\sum \vec{F} = m \vec{a}$
\end{flashcard}
\begin{flashcard}{Newton's second law (for statics)}
$\sum \vec{F} = 0$ when not accelerating
\end{flashcard}
\begin{flashcard}{Newton's third law}
For every action, there is an equal and opposite reaction
\end{flashcard}

\begin{flashcard}{momentum}
$\vec{p} = m \vec{v}$
\end{flashcard}
\begin{flashcard}{$\vec{p}$ units}
\si{\kilo\gram\meter\per\second} or \si{\newton\second}
\end{flashcard}
\begin{flashcard}{impulse}
$\Delta p = F \Delta t = m \Delta v$
\end{flashcard}
\begin{flashcard}{$\Delta p$ units}
\si{\kilo\gram\meter\per\second} or \si{\newton\second}
\end{flashcard}
\begin{flashcard}{momentum conservation}
$\vec{p}_0 = \vec{p}_f$, no outside forces acting
\end{flashcard}

\begin{flashcard}{work}
$W = F\cdot \Delta x$
\end{flashcard}
\begin{flashcard}{work $W$ units}
$\si{\joule} = \si{\newton\meter}$
\end{flashcard}
\begin{flashcard}{kinetic energy}
$KE = \frac{1}{2} m v^2$
\end{flashcard}
\begin{flashcard}{$KE$ units}
$\si{\joule}$
\end{flashcard}
\begin{flashcard}{gravitational potential energy}
$GPE = mgh$
\end{flashcard}
\begin{flashcard}{$GPE$ units}
$\si{\joule}$
\end{flashcard}

\begin{flashcard}{constant velocity kinematics}
$x = v_0t+x_0$, $v=v_0$, $a=0$
\end{flashcard}
\begin{flashcard}{constant acceleration kinematics}
$x = \frac{1}{2}at^2+v_0t+x_0$, $v=at+v_0$, $a\neq 0$
\end{flashcard}
\begin{flashcard}{position during constant velocity motion}
linear, $x=vt+x_0$
\end{flashcard}
\begin{flashcard}{position during constant acceleration motion}
parabola, $x = \frac{1}{2}at^2+v_0t+x_0$
\end{flashcard}
\begin{flashcard}{velocity during constant velocity motion}
constant, $v=v_0$
\end{flashcard}
\begin{flashcard}{velocity during constant acceleration motion}
linear, $v = at+v_0$
\end{flashcard}

\begin{flashcard}{what to put on graphs}
label axes, include numbers and units. graph breakpoints or describe shape of graphs. 
\end{flashcard}
\begin{flashcard}{what to put in problem solutions}
READ THE PROBLEM. write general approach, main equations, then fill in values, then compute answer. sanity check and include units. graph if asked for graphs. 
\end{flashcard}

\begin{flashcard}{Can I do physics?}
YES YOU CAN
\end{flashcard}
\begin{flashcard}{This is hard; why should I do physics if they already figured it out?}
\emph{Ad astra per aspera}, \emph{Nullius in verba}
\end{flashcard}



\end{document}

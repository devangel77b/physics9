\documentclass[exam,addpoints, noanswers]{exam}

\usepackage[exam]{physics9}

\title{Test \#3}
\date{\today}
\author{\mobeardInstructorShort}

\begin{document}
\maketitle
\vfill
\mobeardExamNameBlock
\vfill
Instructions: 
\begin{enumerate}
\item Do not open exam until instructor announces that you may begin.
%\item Please write your name on each page in case the sheets are separated. 
\item Closed notes, closed book.  You may use two \SI{8.5x11}{\inch} note sheets, both sides. 
\item Calculators (TI-84 or equivalent) are permitted.  On standardized tests you will not be permitted to use your cell phone as a calculator, so it is best to get used to your calculator now. 
\item Eight multiple choice are worth five points each, two essay questions (divided into parts) are worth 30 points each.  I have shortened the test (due to today's short period) and shifted more points to the multiple choice based on feedback from the practice test. 
\item In answering the essay questions, be thorough but concise. Deep understanding of the concepts will be displayed by proper use of vocabulary and discussion of the interconnectedness of concepts. 
\item Show your calculations and (most importantly) your \textbf{thinking}.
\end{enumerate}
\vfill
\begin{center}
\gradetable[h][questions]
\end{center}
\clearpage



\begin{questions}
\question[5] Easy warmup question: which of the following is NOT a vector quantity?
\begin{choices}
\CorrectChoice The number of pages in this test
\choice The current course (compass heading) and speed of the nuclear-powered aircraft carrier USS CARL VINSON (CVN-70)
\choice The position of Mr Liese relative to Mr Caldwell
\choice Momentum
\end{choices}



\question[5] Xuan the puppy walks one block east, one block north, one block west, and one block south. What is Xuan's displacement? Please show your work.
\begin{choices}
\choice Zero
\choice Four blocks
\CorrectChoice Zero blocks
\choice Two blocks east
\end{choices}



\question[5] Inspired by the science bear and \emph{Physics for Babies}, Stella Popielski becomes an astronaut and travels into space, where she throws a ball (very far from anything else). What happens to the ball, and why?
\begin{choices}
\choice It eventually comes to a stop because of air resistance
\CorrectChoice Since no outside forces are acting on the ball, it continues moving forever at constant velocity
\choice The ball comes to a stop because every action causes an equal and opposite reaction
\choice Because there are no forces countering the throw, the ball eventually accelerates to beyond the speed of light
\end{choices}



\question[5]When firing the marble shooter, which of the following are strategies you could use to improve the accuracy / reduce the experimental error (and why)? Please select all that apply. 
\begin{choices}
\CorrectChoice Use the same marble, to avoid variation due to marble size and mass
\CorrectChoice Have the same person fire the marble from the same gun, using the same grip every time, to provide as consistent a launch as possible
\choice When collecting marble range data for predictions, use as few data points as possible and use a high order polynomial for the curve fit so that the residuals are zero
\CorrectChoice When collecting marble range data for predictions, collect many data points and randomize the order of angles taken so as to get as true an estimate of the variation as possible. 
\CorrectChoice Fire the marble in the same room, in the absence of large air currents, on a stable floor that is not moving, to try to reduce or control these factors. 
\end{choices}



\question[5] A quadrotor pilot flying a 6S racing quad (mass \SI{0.672}{\kilo\gram}) flies up at \SI{10}{\meter\per\second} for \SI{5}{\second}, hovers for \SI{5}{\second}, then flies down at \SI{10}{\meter\per\second} for \SI{4.9}{\second}. In the space provided, please sketch the vertical velocity and vertical position of the quadrotor versus time during the whole maneuver. 



\question[5] You are checking your friend's physics homework. Without checking their calculations, which answers are most definitely WRONG? Please select all that, without a doubt, have something wrong with them
\begin{choices}
\CorrectChoice velocity = \SI{1.2}{\meter} in positive $x$ direction
\CorrectChoice acceleration = \SI{9.81}{\meter\per\second} down
\CorrectChoice $x = 5$ 
\CorrectChoice velocity = \SI{3e22}{\meter\per\second} right
\CorrectChoice distance = \num{-32767}
\choice force = \SI{1500}{lbf}, in tension
\choice mass = \num{7840} long tons
\choice speed = \num{42} furlongs/fortnight
\CorrectChoice Because of Newton's fourth law
\CorrectChoice The observations are wrong, because Aristotlean philosophy says the bocce ball and the basketball should fall at different rates, therefore it is unnecessary to even make the measurement. 
\end{choices}




\question[5] Dr Evangelista is driving at \SI{10}{\meter\per\second} when he sees a cat \SI{100}{\meter} in front of him. He immediately hits the brakes and begins decelerating at \SI{10}{\meter\per\second\squared}. How long does it take for him to come to a stop and why? 
\begin{choices}
\choice \SI{1}{\second}, using $x = \dfrac{1}{2} a t^2 + v_0 t + x_0$
\CorrectChoice \SI{1}{\second}, using $v = a t + v_0$
\choice \SI{-50}{\meter}, using $x = \dfrac{1}{2} x_0 t^2 + v_0 t + x_0$
\choice It is impossible to determine how long it takes him to stop without knowing $x_0$
\end{choices}




\question[5]According to Wikipedia, the Electromagnetic Aircraft Launching System (EMALS) is 300 feet (\SI{91}{\meter}) long and can accelerate a 100,000 pound (\SI{45000}{\kilogram}) aircraft to 130 knots (\SI{67}{\meter\per\second}) in \SIrange{2}{3}{\second}. In SI units, what is the maximum acceleration? Please show your work. 
\begin{choices}
\CorrectChoice \SI{33}{\meter\per\second\squared}
\choice \SI{22}{\meter\per\second\squared}
\choice \SI{150}{\meter\per\second\squared}
\choice \SI{66}{\meter\per\second\squared}
\end{choices}




\clearpage
\question  An MBS theater student is cast at the next super spy Jane Bond (reboot). 
\begin{parts}
\part[5] As part of a dramatic scene in which an evil villian has her cornered on a zeppelin, Jane Bond steps out of the zeppelin and begins falling. Draw a free body diagram of Jane Bond at this point, before she has reached a speed at which air resistance is significant. 

\part[5] During the first part of the fall, what type of motion is Jane Bond undergoing? 
\begin{choices}
\choice constant velocity motion
\choice simple harmonic motion
\CorrectChoice motion with constant acceleration at \SI{-9.81}{\meter\per\second\squared}
\choice motion with constant acceleration at \SI{-32}{\meter\per\second\per\second}
\choice elliptical orbit
\end{choices}

\part[5] Which equations apply during the first part of Jane's fall? Please select all that apply
\begin{choices}
\CorrectChoice $y = \dfrac{1}{2} a t^2 + v_0 t + y_0$
\CorrectChoice $v = at + v_0$
\choice $y = vt + y_0$
\choice $v = \text{constant}$
\choice $y = Y \sin(\omega t)$
\end{choices}

\part[5] Continuing, the first part of the Jane Bond shot lasts \SI{4}{\second}. To aid in planning where to put the cameras, please compute how far she falls during this time. Please show your work. 

\part[5] Jane Bond then triggers her Apple Watch which causes her tuxedo to deploy a patagium membrane wing suit that slows her fall, causing her to reach terminal velocity (where air resistance balances weight). Please draw a free body diagram for this new situation with the wing suit deployed. 

\part[5] During the second part of the fall, with the wing suit deployed, what type of motion does Jane undergo? 
\begin{choices}
\CorrectChoice Constant velocity motion
\choice Simple harmonic motion
\choice Elliptical orbit
\choice Phugoid gliding
\choice Dutch roll and short period oscillation
\end{choices}
\end{parts}




\clearpage
\question For the second short answer question, you get a job at NASA working on a small, zero G robot propelled with compressed air, able to move around inside the International Space Station. You are planning a set of commands to transmit to the robot for an upcoming flight test on a ``Vomit Comet'' C-9B microgravity simulator plane. 

\begin{parts}
\part[10] The first test maneuver is to fly up \SI{2}{\meter} in \SI{8}{\second}, hover for 
\SI{8}{\second}, then return to the launch point in \SI{8}{\second}. In the space here, sketch the vertical position and velocity of the robot. (You will need this to construct velocity commands to send to the robot, and to set failsafe limits on position in its autonomous flight controller.) 

\part[10] For the same maneuver, sketch the acceleration of the robot. Please ignore the instants at \SIlist{0; 8; 16; 24}{\second}. (If you like a challenge feel free to guess what happens around these points.)

\part[10] During the actual test, a failure occurs in the propulsion system, resulting in the throttle valve being stuck open during the first part of the maneuver (heading upwards). With the failed valve, the force on the robot is \SI{10}{\newton} and its mass is \SI{1}{\kilo\gram}. Please plot the robot's position versus time. The far wall of the test chamber is \SI{10}{\meter} away; show when it hits the wall on your graph. Assume the robot is streamlined so that air resistance is small and neglect any change in mass from the expulsion of propellant.  
\end{parts}
\end{questions}
\end{document}
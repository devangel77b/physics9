\documentclass[quiz,addpoints,noanswers]{exam}

\usepackage[quiz]{../physics9}

\title{Quiz: Energy (intro)}
\date{\today}
\author{\mobeardInstructorShort}

\begin{document}
\maketitle
\begin{abstract}
To help with your problem solving skills and build confidence in applying physics to actual situations we will start having more quizzes; this will also help even out the grading. The quizzes are meant to be low stress and short; you may use your notes to help recall concepts you may have missed but try your best to answer without lookups. You should not need to spend more than \SI{5}{\min} on this quiz. 
\end{abstract}

\begin{questions}
\question[1] Which of the following are legit for real forms of energy? Please mark all that apply. One you could probably argue either way.
\begin{choices}
\CorrectChoice gravitational potential
\CorrectChoice kinetic
\CorrectChoice elastic stored 
\choice momentum
\choice velocity
\choice Monster Drink
\CorrectChoice chemical 
\CorrectChoice nuclear
\end{choices}

\question[2] Xuan the Guide Dog Puppy (mass \SI{65}{\pound}, \SI{29.4}{\kilo\gram}) is in the park when he is given the recall command \textbf{come!}. He begins running back towards his handler at \SI{8}{\meter\per\second}. Compute Xuan's kinetic energy.  
\begin{solution}[1in]
The equation you need is $KE=\frac{1}{2}mv^2$. $KE=\frac{1}{2}mv^2=0.5\cdot\SI{29.4}{\kilo\gram}\cdot\SI{8}{\meter\per\second}^2$. The final answer is \SI{941}{\joule}
\end{solution}

\question[2] Cat (mass \SI{12}{\pound} or \SI{5.4}{\kilo\gram}) is on the counter (height \SI{0.91}{\meter}) looking for yogurt, when she steps off the edge and lands on the floor. What is her kinetic energy at landing?
\begin{solution}
Cat's gravitational potential energy is $GPE=mgh$ or $\SI{5.4}{\kilo\gram}\cdot\SI{9.81}{\meter\per\second\squared}\cdot\SI{0.91}{\meter}$, so $GPE=\SI{48.2}{\joule}$. Because the fall is short and the speed is relatively low, we will neglect drag. Because of energy conservation, her final kinetic energy at ground level is equal to her initial gravitational potential energy, therefore $\text{final}\ KE=\SI{48.2}{\joule}$. 
\end{solution} 

\end{questions}
\end{document}
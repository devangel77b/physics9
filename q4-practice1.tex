\documentclass[exam,addpoints, noanswers]{exam}

\usepackage[exam]{physics9}

\title{Practice Test \#4A}
\date{\today}
\author{\mobeardInstructorShort}

\begin{document}
\maketitle
\vfill
\mobeardExamNameBlock
\vfill
Instructions: 
\begin{enumerate}
\item Do not open exam until instructor announces that you may begin.
\item Please write your name on each page in case the sheets are separated. 
\item Closed notes, closed book.  You may use two \SI{8.5x11}{\inch} note hardcopy sheets, both sides. You may NOT use your computer or phone as your notes. 
\item Calculators (TI-84 or equivalent) are permitted.  On standardized tests you will not be permitted to use your cell phone as a calculator, so it is best to get used to your calculator now. You may NOT use your computer or phone as your calculator. 
\item Eight multiple choice are worth five points each, three essay questions (divided into parts) are worth 20 points each. 
\item In answering the essay questions, be thorough but concise. Deep understanding of the concepts will be displayed by proper use of vocabulary and discussion of the interconnectedness of concepts. 
\item Show your calculations and (most importantly) your \textbf{thinking}.
\end{enumerate}
\vfill
\begin{center}
\gradetable[h][questions]
\end{center}
\clearpage



\begin{questions}
\question[5] Easy warmup question: what are the correct SI units for velocity? 
\begin{choices}
\CorrectChoice \si{\meter\per\second}
\choice \si{\per\meter\per\second}
\choice \si{\foot\per\second}
\choice \si{\kilo\meter\per\hour}
\end{choices}

\question[5] For a Hollywood stunt in the next Jurassic Park film, a \emph{\dag Triceratops} will stop a humvee head on. To design the rig for the shot, assume the humvee has mass $m=\SI{3500}{\kilo\gram}$ and is traveling at 25 mph ($v=\SI{11}{\meter\per\second}$). What is the change in momentum of the humvee, assuming the \emph{\dag Triceratops} is at rest. 
\begin{choices}
\choice Zero, since momentum is conserved. 
\choice \SI{38500}{\meter\per\second}
\CorrectChoice \SI{38500}{\kilo\gram\meter\per\second}
\choice \SI{962500}{\meter\per\second}
\end{choices}

\question[5] Which of the following is a correct mathematical expression for impulse. Please select all that apply. 
\begin{choices}
\CorrectChoice $F \Delta t$
\CorrectChoice $m \Delta v$
\CorrectChoice $\Delta p$
\choice $\frac{1}{2} m v^2$
\choice $m g h$
\end{choices}

\question[5] Which of the following are situations in which you might expect momentum to be conserved?
\begin{choices}
\CorrectChoice An inelastic collision between two carts with no outside forces acting
\CorrectChoice An elastic collision between two carts with no outside forces acting
\CorrectChoice An explosion between two carts with no outside forces acting
\choice A yellow labrador retriever accelerating as he runs to catch a ball
\CorrectChoice An airborne labrador retriever at the instant of catching a ball, neglecting air resistance. 
\end{choices}

\clearpage
\question[5] Which do you expect to have the most momentum?
\begin{choices}
\choice A stationary blue whale
\CorrectChoice A blue whale freefalling from a plane at \SI{34}{\meter\per\second}
\choice A mouse freefalling from a plane at \SI{34}{\meter\per\second}
\choice A stationary mouse
\choice A mouse running at \SI{1}{\meter\per\second}
\end{choices}

\question[5] A cue ball (the white one) is moving at \SI{1}{\meter\per\second} towards an eight ball (the black and white one). Assuming the masses are equal, the collision is a perfectly elastic collision, and that the cue ball strikes the eight ball perfectly head on, the final speed of the eight ball is approximately what?
\begin{choices}
\CorrectChoice \SI{1}{\meter\per\second}
\choice \SI{0.5}{\meter\per\second}
\choice \SI{2}{\meter\per\second}
\choice \SI{10}{\meter\per\second}
\choice \SI{-1}{\meter\per\second}
\end{choices}

\question[5] Elmer Fudd fires a gigantic artillery shell at Bugs Bunny, but through Bugs' clever use a rotating floor, Elmer ends up eating the artillery shell. The collision between Elmer and the shell is best described as a(n)
\begin{choices}
\CorrectChoice inelastic collision
\choice elastic collision
\choice explosion
\choice deep impact
\choice event horizon
\end{choices}

\question[5] The tallest roller coaster at Six Flags Great Adventure is the Kingda Ka (elevation \SI{456}{\foot} (\SI{139}{\meter}), max speed 128 mph (\SI{57}{\meter\per\second}). What physics conservation principle would be most useful in understanding the relationship between elevation and max speed?
\begin{choices}
\choice conservation of linear momentum
\choice conservation of charge
\choice conservation of mass
\choice conservation or angular momentum
\CorrectChoice conservation of energy
\end{choices}





\clearpage
\question  A cue ball (the white one, $m=\SI{170}{\gram}$) is moving at \SI{1}{\meter\per\second} straight towards an eight ball (the black and white one, also $m=\SI{170}{\gram}$) initially at rest.
\begin{parts}
\part[5] List some reasonable assumptions and a general approach you would use to find the final velocity of the eight ball.
\begin{solution}[1in]
I assume the cue ball hits the eight ball straight on and the collision is perfectly elastic. The general approach is momentum conservation for a perfectly elastic collision. 
\end{solution}
\part[5]\label{part:9b} Write the governing equation(s) for the collision.
\begin{solution}[1in]
\begin{align*}
m_1 v_{1,0} + m_2 v_{2,0} &= m_1 v_{1,f} + m_2 v_{2,f}\ \text{momentum conservation} \\
\frac{1}{2} m_1 v_{1,0}^2 + \frac{1}{2} m_2 v_{2,0}^2 &= \frac{1}{2} m_1 v_{1,f}^2 + \frac{1}{2} m_2 v_{2,f}^2\ \text{energy conservation}
\end{align*}
\end{solution}
\part[5] Write the numerical values for the known terms of your equation from part~\ref{part:9b}, e.g. $m_1$, $m_2$, $v_{1,0}$, $v_{2,0}$. 
\begin{solution}[1in]
\begin{align*}
m_1 = \SI{0.170}{\kilo\gram} \\
m_2 = \SI{0.170}{\kilo\gram} \\
v_{1,0} = \SI{1}{\meter\per\second} \\
v_{2,0} = \SI{0}{\meter\per\second} 
\end{align*}
\end{solution}
\part[5] Find the final velocity of the eight ball. 
\begin{solution}[1in]
$v_{2,f}=\SI{1}{\meter\per\second}$. 
\end{solution}
\end{parts}

\clearpage
\question
\begin{parts}
\part[7] A blue whale (\emph{Balaenoptera musculus}, mass $m=\SI{330000}{\pound}$) is launched by a linear induction motor at \SI{34}{\meter\per\second}. Compute its momentum. ($\SI{2.2}{\pound}=\SI{1}{\kilo\gram}$). 
\begin{solution}[1in]
$p = \SI{5.1e6}{\kilo\gram\meter\per\second}$
\end{solution}
\part[7] A mouse (\emph{Mus musculus}, mass $m=\SI{0.68}{oz}$) wishes to have the same momentum (but opposite) as the blue whale. It is launched by a linear induction motor in the opposite direction. What speed must the mouse have to have the same magnitude of momentum? ($\SI{1}{oz}=\SI{0.02835}{\kilo\gram}$)
\begin{solution}[1in]
$v=\SI{2.6e8}{\meter\per\second}$ (not a very likely speed!) 
\end{solution}
\part[6] The two collide in an inelastic collision. What is the final speed of the whale-mouse system? 
\begin{solution}[1in]
$v_f=\SI{0}{\meter\per\second}$
\end{solution}
\end{parts}

\clearpage
\question Christina Ricci (former MBS student!) is preparing to film a new action movie reboot of the Addams Family. As part of the movie, she plans to jump from an exploding helicopter onto the MBS football field. Your job is to evaluate two stunt coordinators and decide which one to hire for the shot.
\begin{parts}
\part[10] The first candidate suggests using as rigid, tough, and massive a material as possible (astroturf on top of concrete) for Ms Ricci to land on, explaining that since $m$ is large, $\Delta v$ will be small and as a result, there will be minimal chance of injury. Please comment. Should we hire this one? 
\begin{solution}[2.5in]
\end{solution}
\part[10] The second candidate suggests carefully stacking a large number of cardboard boxes as cushioning, and claims that by extending the landing over a long time $\Delta t$, the force $F$ on Ms Ricci will be small and there will be minimal chance of injury. Please comment. Should we hire this one? 
\begin{solution}[2.5in]
\end{solution}
\end{parts}
\end{questions}
\end{document}
\documentclass [hw]{exam}

\usepackage{physics9}

\title{Project \#3: In-class design warmup}
\author{\mobeardInstructorShort}
\date{\today}
\duedate{in class}

\begin{document}
\maketitle

\begin{questions}
\question Pair up. You may work with a friend but if you choose a friend you end up socializing with rather than working, that's your own fault. 

\question Your instructor has assembled three stations with a consumer product. Pick one, individually sketch the different products you see (like on paper) and evaluate if the products meet the needs they were designed to fill. This is a timed exercise, take \SI{10}{\minute}.

\question Working individually, brainstorm things your device must do. Try NOT to think of solutions at this point; just identify the things you device must do. Write each on a single yellow stickie note, preferably of the form ``The device must do X (when Y happens)''; or ``The user experiences A'' or similar. This is a timed exercise, take \SI{10}{\minute}.

\question Now combine yours with your partners and sort the stickies. Use blue stickies to group the needs and functional requirements you have identified. See if you can rank the functional requirements from highest (weight 3) to lowest (weight 1) based on importance. 

\question Again working individually, brainstorm ways to accomplish the functional requirements you have identified. This is a timed exercise, take \SI{10}{\minute}. For each idea, do a Pictionary-grade sketch and some descriptive words. 

\question Swap your ideas with your partner and review them. Write two pieces of positive feedback and two pieces of negative for each. Take \SI{2.5}{\minute}

\question Again working individually, brainstorm new ways to accomplish the functional requirements you have identified. This is a timed exercise, take \SI{10}{\minute}. For each idea, do a Pictionary-grade sketch and some descriptive words. You can riff off your partner's, combine ideas, etc. 

\question Swap your ideas with your partner and review them. Write two pieces of positive feedback and two pieces of negative for each. Take \SI{2.5}{\minute}

\question Now, discuss with your partner how to move forward. Are there test you want to perform? Is there benchmarking you need to do to look at similar devices? Are there prototypes you need to build and try before deciding on a final concept design? 

\question Update your (individual) Engineering Design Notebooks to include what you have worked on today.
\end{questions}
\end{document}
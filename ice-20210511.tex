\documentclass [hw,addpoints,noanswers]{exam}

\usepackage[hw]{physics9}

\title{In-class exercise: Three view drawings}
\author{\mobeardInstructorShort}
\date{\today}
\duedate{in class}

\begin{document}
\maketitle

\begin{abstract}
Engineering drawings are a standard way to communicate your design ideas with other scientists and engineers. Your instructor will show you an example. Based on this, prepare some drawings of your device for inclusion in your design notebook. They don't have to be pretty or even perfectly to scale; just do your best to be neat and clearly communicate what your design is.  You may also wish to arrange these on the page in a standard fashion, centered and balanced. 
\end{abstract}

\begin{questions}
\question[4] Draw a \textbf{front view} of your device. 
\question[4] Draw a \textbf{side view} of your device.
\question[4] Draw a \textbf{top view} of your device. 
\question[4] Draw a \textbf{isometric view} of your device. 
\question[4] Include some \textbf{dimensions} and \textbf{callouts} 
\end{questions}
\end{document}
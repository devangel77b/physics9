\documentclass[exam,addpoints, answers]{exam}

\usepackage[exam]{physics9}

\title{Practice Test \#3A}
\date{\today}
\author{\mobeardInstructorShort}

\begin{document}
\maketitle
\vfill
\mobeardExamNameBlock
\vfill
Instructions: 
\begin{enumerate}
\item Do not open exam until instructor announces that you may begin.
\item Closed notes, closed book.  You may use two \SI{8.5x11}{\inch} note sheets, both sides. 
\item Calculators are permitted, but show your work in order to receive partial credit. 
\item Please write your name on each page in case the sheets are separated. 
\end{enumerate}
\vfill
\begin{center}
\gradetable[h][questions]
\end{center}
\clearpage

\begin{questions}
\question[1] Which of the following is \textbf{NOT} a vector quantity?
\begin{choices}
\choice position
\choice velocity
\choice acceleration
\CorrectChoice distance
\end{choices}

\question[1] Xuan the puppy walks one block east, one block north, one block east, and one block south. What is Xuan's displacement? Please show your work; graph Xuan's movement in the space here. 
\begin{choices}
\CorrectChoice Two blocks east
\choice Four blocks.
\choice Two.
\choice Two blocks east, two blocks north. 
\end{choices}

\question[1] Xuan the puppy walks one block east, one block north, one block east, and one block south. What distance did Xuan travel? Please show your work. 
\begin{choices}
\choice Two blocks.
\CorrectChoice Four blocks.
\choice Two blocks east.
\choice Four. 
\end{choices}

\question[1] A rock on the ground is unlikely to spontaneously move \SI{2}{\meter} to the right. This observation corresponds to which physical law? 
\begin{choices}
\CorrectChoice Newton's first law.
\choice Ohm's law. 
\choice Kirchoff's current law. 
\choice Kirchoff's voltage law. 
\end{choices}
\begin{solution}
Remember there may be some ``noise'' answers to try to confuse you -- don't be fooled. 
\end{solution}

\question[1] A rocket in space fires its thrusters, causing it to turn. Which of Newton's laws apply?  Circle all that apply. 
\begin{choices}
\CorrectChoice Newton's first law. 
\CorrectChoice Newton's second law.
\CorrectChoice Newton's third law. 
\choice None of the above.
\end{choices}




\clearpage
\question[1] Which force(s) are most important to consider in a bocce ball dropped from the second floor balcony? Circle all that apply and draw and applicable free body diagram. 
\begin{choices}
\CorrectChoice Weight or gravity. 
\choice Air resistance, drag, or other aerodynamic forces. 
\choice Van der Waals force. 
\choice None of the above.
\end{choices}
\begin{solution}
The bocce ball fell in time indistinguishable (to our instrument precision) from the basketball, softball, etc. Air resistance could be modeled, but its effect is small (compared to the paper) such that we could leave it out for a simpler analysis that still works. In physics we like the simplest model that describes what we see,. 
\end{solution}

\question[1] Which force(s) are most important to consider in a uncrumpled leaf of paper dropped from the second floor balcony? Circle all that apply and draw an applicable free body diagram. 
\begin{choices}
\CorrectChoice Weight or gravity. 
\CorrectChoice Air resistance, drag, or other aerodynamic forces. 
\choice Van der Waals force. 
\choice None of the above.
\end{choices}
\begin{solution}
In the falling paper, the drag forces were significant and resulted in a much slower fall than all the other things we dropped. We cannot neglect air resistance in this case. 
\end{solution}

\question[1] In 2D projectile motion, why is the horizontal speed more or less constant?
\begin{choices}
\CorrectChoice Because the sum of the forces in the $x$-direction (neglecting air resistance) is zero. 
\choice Because constant velocity motion is all we know how to analyze. 
\choice Because the action of any air resistance causes an equal an opposite reaction from the gun/propellant. 
\choice None of the above. 
\end{choices}

\question[1] When firing the marble gun with the \textbf{exact} same settings, the ranges observed are \SIlist{1.1; 1.09; 1.07; 1.12; 1.2}{\meter}. The most likely explanation for this is: 
\begin{choices}
\CorrectChoice Experimental error caused by small variations in the launch process, air currents, etc. that are difficult to control.  
\choice Use of different marbles, different people launching it, and different marble launchers. 
\choice Marble range is not predictable using science, observation, experiment, or modeling. 
\choice The action of fairies in the aether. 
\end{choices}
\begin{solution}
Careful groups that tried to control as many factors as possible tried to avoid using different marbles, different people launching them, or different launchers; the other factors in A were much more difficult to control even for very careful groups. 
\end{solution}

\question[1] Which equation best models the movement of a point mass under uniform acceleration?
\begin{choices}
\CorrectChoice $x(t) = \dfrac{1}{2} a t^2 + v_0 t + x_0$
\choice $x(t) = \dfrac{1}{2} a t^2$
\choice $x(t) = v t$
\choice $x(t) = x_f \left( 1 - e^{-\frac{t}{\tau}} \right)$
\end{choices}






\clearpage
\question[23] Dr.~Evangelista is driving at \SI{10}{\meter\per\second} when he sees a cat \SI{100}{\meter} in front of him. He immediately hits the brakes and begins decelerating at \SI{10}{\meter\per\second\squared}. Does he hit the cat? 
\begin{solution}[6in]
First consider the speed of the truck. The initial speed $v_0=\SI{10}{\meter\per\second}$ and the truck is decelerating at $a=\SI{-10}{\meter\per\second\squared}$. The equation for speed is then 
\begin{align}
v &= at + v_0 \\
&= (-10) t + (10) 
\end{align}
Solving for where $v=0$ (when the truck comes to a stop) gives $t=\dfrac{\SI{-10}{\meter\per\second}}{\SI{-10}{\meter\per\second\squared}}=\SI{1}{\second}$. 

We can then plug this in to find the truck's displacement when it comes to a stop:
\begin{align}
x &= \dfrac{1}{2}at^2 + v_0 t + x_0 \\
&= \dfrac{1}{2}(\SI{-10}{\meter\per\second\squared})(\SI{1}{\second})^2 + (\SI{10}{\meter\per\second})(\SI{1}{\second}) + 0 \\
&= \SI{-5}{\meter} + \SI{10}{\meter} \\
&= \SI{5}{\meter}.
\end{align} 

Dr Evangelista does not hit the cat. 
\end{solution}

\clearpage
\question[23] Xuan the puppy chases his tail, resulting in him running in a small diameter circle. In the space provided, explain to Xuan the difference between his speed, the distance he travels, and his displacement. 
\begin{solution}[6in]
The speed the puppy chases his tail at is however fast he is running in circles. 

The \emph{distance} he travels is however many turns he makes, but his \emph{displacement} is zero or very small since he is essentially spinning in place. 
\end{solution}

\clearpage
\question[22] While training in a tank to simulate zero gravity, a NASA astronaut draws a free body diagram of herself as she pushes horizontally on a simulated satellite. Assume the buoyancy of the water is equal to her weight. Draw and explain what you think the free body diagram looks like. 
\begin{solution}[6in]
In the vertical direction, there is a buoyancy force pointing upwards equal to her weight, and her weight pointing downwards. In the horizontal direction, there is the unbalanced reaction force from pushing on the satellite; she will begin accelerating unless she is able to brace on the satellite or on a robot arm or EVA suit. In the tank, which is in water, there may also be drag and added mass (fluid) forces resisting horizontal acceleration. See sketch. 
\end{solution}

\clearpage
\question A quadrotor pilot is flying a 6S racing quad with mass \SI{0.672}{\kilo\gram}. 
\begin{parts}
\part[11] Draw a free body diagram of the quadrotor while it is hovering. 
\begin{solution}[3in]
While (steadily) hovering, the forces acting are the weight of the quadrotor (gravity) and the thrust from the four rotors. 

\end{solution}
\part[11] She increases the throttle so that the quadrotor accelerates upwards at 8$g$. Assuming it started at $y=0$, plot $y(t)$, the height of the quadrotor.  Draw a free body diagram for the quadrotor during this maneuver. 
\begin{solution}[3in]
During the vertical maneuver, the thrust of the rotors is 9x the weight of the quadrotor, so that the net force on the quad rotor is 8x the weight. This results in an 8$g$ acceleration. 

The height of the quadrotor would go as $y=\dfrac{1}{2}(8\cdot\SI{9.81}{\meter\per\second\squared})t^2$. 
\end{solution}
\end{parts}
\end{questions}
\end{document}
\documentclass[quiz,addpoints,noanswers]{exam}

\usepackage[quiz]{../physics9}

\title{Quiz: Momentum -- Collisions and design of an experiment!}
\date{\today}
\author{\mobeardInstructorShort}

\begin{document}
\maketitle

\begin{abstract}
Let us consider two-body collisions, in which two bodies collide and either stick together (\textbf{inelastic}, as in your recent homework) or bounce off each other (\textbf{elastic}). We may also consider a case where two bodies initially stuck together fly apart (\textbf{explosion}).  \textbf{This is a for-real quiz.} You will have all class period to work on it. You will \textbf{not} get an automatic 5/5 on this one, you have to earn it, it is \textbf{graded on correctness (in that you really develop a testable hypothesis and devise an experiment to test it) not completion}. Please answer with full sentences, equations, or clear sketches. 
\end{abstract}



\begin{questions}
\question[2] Develop a \textbf{testable hypothesis} about what you think the momentum and energy of the system (both bodies together) will do. A testable hypothesis is one in which you could do an experiment and the results could somehow show the hypothesis is not true. Express your testable hypothesis both in words, and in equations if possible. For brevity, you may choose to deal with only one type of collision (inelastic, elastic, or explosion). Do you have any background knowledge that informs your choice of hypothesis? Is there an alternative hypothesis? 
\begin{solution}[4in]
\end{solution}

\clearpage
\question[2] Propose an \textbf{experiment} that could be done within a class period to test your hypothesis. Describe what methods you might use, and explain any expected problems, or special materials or equipment you think you might need.
\begin{solution}[3in]
\end{solution}

\question[1] List what \textbf{data} you wish to gather. If necessary, provide a drawing and label or list the quantities you wish to measure (e.g. $m$, $v$, etc), when you want to measure them, and how many \textbf{replicates} you wish to do.  If you must plan a series of measurements with some independent variable changing, please list what you wish to measure.
\begin{solution}[3in]
\end{solution}

\end{questions}
\end{document}
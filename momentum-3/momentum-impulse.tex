\documentclass[handout]{tufte-handout}

\newcommand{\myroot}{..}
\usepackage[handout]{\myroot/physics9}
\usepackage{textcomp}

\title{Momentum -- impulse}
\date{\today}
\author{\mobeardInstructorShort}

\begin{document}
\maketitle
\marginnote{TLDR: The equation for impulse is $\Delta p = F \Delta t = m \Delta v$.}
\marginnote{A good habit to pick up is to actually read the textbook or the notes.}

\section{Introduction}
Recall Newton's second law, which we wrote in simplified form as 
\begin{equation}
\sum \vec{F} = m \vec{a}.
\label{eq:secondlaw}
\end{equation}
Also recall the definition of acceleration
\begin{equation}
\text{acceleration}\ a = \dfrac{\text{change in velocity}}{\text{change in time}} = \dfrac{\Delta v}{\Delta t}.
\label{eq:acceleration}
\end{equation}
Combining \fref{eq:secondlaw} and \fref{eq:acceleration} gives
\begin{equation}
F = m \dfrac{\Delta v}{\Delta t},
\end{equation}
which we can rearrange by multiplying both sides by $\Delta t$:
\begin{equation}
F \Delta t = m \Delta v.
\label{eq:impulse1}
\end{equation}

The terms on both sides of \fref{eq:impulse1} have units of \si{\newton\second} or \si{\kilo\gram\meter\per\second}, which are both units of momentum. You might remember we introduced these as \textbf{impulse} when we introduced momentum\sidenote{Here we drop the vector symbols because in \mobeardCourseName, we will probably only consider one-dimensional problem.}:
\begin{equation}
\Delta p = F \Delta t = m \Delta v. 
\label{eq:impulse}
\end{equation}

\subsection{Uses of impulse}
Impulse as given in \fref{eq:impulse} is quite useful when you have a device (such as a rocket motor) that puts out some amount of force over a fixed amount of time. Often, what we care most about is the final velocity, and \fref{eq:impulse} gives a very straightforward way to compute it. 

For example, model rocket motors (\fref{tab:rocketmotors}) are classified and purchased based on the impulse they produce. When purchasing a rocket motor, the manufacturer will specify the impulse, the average force, and the delay between the end of the thrust phase and the ignition of any ejection charges. 
\begin{margintable}
\caption{Model rocket motor size classes. Do not memorize this table or put it on your notes, if you need information like this on a test you it will be given to you.}
\label{tab:rocketmotors}
\begin{center}
\begin{tabular}{rr}
\toprule
class & impulse, \si{\newton\second} \\
\midrule
\textonequarter A & \numrange{0.313}{0.625} \\
\textonehalf A & \numrange{0.626}{1.25} \\
A & \numrange{1.26}{2.50} \\
B & \numrange{2.51}{5.0} \\
C & \numrange{5.01}{10} \\
D & \numrange{10.01}{20}\\
E & \numrange{20.01}{40}\\
F & \numrange{40.01}{80}\\
G & \numrange{80.01}{160}\\
\bottomrule
\end{tabular}
\end{center}
\end{margintable} 

In full-scale space propulsion, impulse provides a way to quickly calculate how long a thruster should be fired given some desired change in velocity. 

Impulse may also be used in considering the average force on a structure from a time varying (such as a periodic) load. 

\clearpage
\section{Example problems}
Clear hints to use impulse, i.e. \fref{eq:impulse}, in a problem include when the impulse is mentioned, or when you have three out of four of: force, time, mass, or velocity. These often show up in the context of cars accelerating or decelerating, football or hockey players or balls or pucks being hit by impulses, etc. 

As you learn about concepts in physics, see if you can think of example questions on your own that stretch your ability to apply the concepts. 

\subsection{Example: a car accelerating from rest at a stoplight}
In a typical problem of this type, a car of mass $m$ is waiting at a stoplight. The light turns green, and the car experiences a force $F$ over some time $\Delta t$ that might be from the driver hitting the gas pedal for $\Delta t$ seconds. What is the car's final velocity? 

You could do this using Newton's second law ($F=ma$) and kinematics:
\begin{align}
a &= \dfrac{F}{m} \\
v &= a t + \cancelto{0}{v_0} \\
&= \dfrac{F}{m} \Delta t \\
x &= \dfrac{1}{2} a t^2 + \cancelto{0}{v_0} t + \cancelto{0}{x_0} \\
&= \dfrac{1}{2} \dfrac{F}{m} (\Delta t)^2
\end{align}
by finding the acceleration, then finding the velocity, and, optionally, the position. This takes several steps. 

A \textbf{shorter way}, if the position is not required, is to use \textbf{impulse}, i.e. \fref{eq:impulse}:
\begin{align}
\text{impulse equation}\ \dfrac{1}{\cancel{m}} \cancel{m} \Delta v &= \dfrac{1}{m} F \Delta t \\
\Delta v &= \dfrac{F}{m} \Delta t.
\end{align}
The benefit is it gives us the answer in one step, which is \textbf{faster} and less prone to errors.  Write out your approach, then fill in the numbers, then make sure your answer makes sense and has the right units. 

\subsection{Example: a car braking}
A common variation is to consider the car during braking. A car of mass $m$ is traveling at speed $v$. In order to not hit Bambi, the car must stop in time $\Delta t$. What average braking force is required? 
\begin{align}
F \Delta t = m \Delta v \\
F = m \dfrac{\Delta v}{\Delta t}.
\end{align}
Note this could also be obtained through the definition of acceleration. If you choose to do such a problem using impulse and momentum, note that you are simply re-arranging \fref{eq:impulse} as necessary to get the unknown quantity on one side and the known quantities on the other. 
 

\subsection{Examples in sports}
Sports related problems often center around a sportsperson of mass $m$ who experiences either an impulse $\Delta p$ or a certain force $F$ over a time $\Delta t$. The problem might ask to find the change in velocity.
\begin{align}
\Delta p = F \Delta t = m \Delta v \\
\Delta v = \dfrac{1}{m} \Delta p = \dfrac{F}{m} \Delta t.
\end{align}

You might be tempted to write down all these different formulas... I suggest you \textbf{don't}; instead try to see them as manipulations of \fref{eq:impulse} that you can do on the fly quickly, This way you have less stuff to write down, look up, or get confused; instead use the base concept and your brain to adapt it as necessary. 



\subsection{Rocket example}
You wish to make a functioning ejection seat for a Buzz Lightyear action figure of mass $m$. Select a size of rocket motor from \fref{tab:rocketmotors} and specify how many are needed to achieve an ejection velocity of $v$ from rest (a so-called ``zero-zero'' ejection seat design). 

The impulse required is $\Delta p = m \Delta v$. Use the table data to select a rocket motor(s) in the required range. 

\end{document}
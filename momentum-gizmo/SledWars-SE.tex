Name:
	

	Date:
	

	

Student Exploration: Sled Wars


Directions: Follow the instructions to go through the simulation. Respond to the questions and prompts in the orange boxes.


Vocabulary: acceleration, energy, friction, kinetic energy, momentum, potential energy, speed
 SledWarsSE1 

Prior Knowledge Questions (Do these BEFORE using the Gizmo.)


1. A firefighter is trying to break through a door. Is he more likely to break through if he runs at the door very slowly or if he runs at the door very fast? Explain your answer.




	

2. Two firefighters are trying to break through a door. One firefighter is heavy, and the other is light. If they run at the same speed, which one is more likely to break through? Explain.




	                                                                                                         SledWarsSE1 
Gizmo Warm-up
The ability to crash through doors and cause other changes depends on an object’s energy. With the Sled Wars Gizmo, you will explore the factors that affect the energy of a sled.


The Gizmo shows a Yeti named Burt riding his sled down a steep hill. Burt plays a game where he tries to wreck as many snowmen as he can. 




2. Use the slider to set the Snowmen to 5. Check that the height of the sled is 25.0 m. Click Play ( Play ). How many snowmen does the sled destroy?
	

	   3. Click Reset ( Reset ). Set the Red sled mass to 200 kg. Click Play. How many snowmen does the sled destroy this time?
	

	

      1. Click Reset. Drag the sled to the top of the hill so the starting height is 50.0 m. Click Play. 


A. How many snowmen are destroyed now?
	

	B. Why do you think this is so?
	

	







Activity A: 


Acceleration and speed
	Get the Gizmo ready: 
      * Click Reset. Check that One sled is selected.
      * Set the Red sled mass to 100 kg.
      * Set the Number of snowmen to 0.
      * Check that the height of the sled is 50.0 m.
	 SledWarsSE2 

	

Introduction: Speed is a measure of how fast an object is moving. It is defined as distance moved per unit of time. In this Gizmo, the units of speed are meters per second, or m/s.


Question: What factors affect the speed of the sled?
      1. Observe: With the sled at 50.0 m, click Play. Observe the Red sled speed speedometer. 


      1. What happens to the sled’s speed as it moves down the slope?
	

	      2. Acceleration is a change in speed or direction over time. In what two ways does the sled accelerate as it descends?
	

	      3. Click Reset. This time, focus on the speed of the sled after it reaches the bottom of the hill. Click Play. What do you notice?
	

	      4. Does the sled accelerate after reaching the bottom? Explain.
	

	

      2. Experiment: Click Reset. Set the Red sled mass to 10 kg. Click Play and then Pause ( Pause ) after the sled reaches the bottom.
         1. Look at the speedometer. What is the speed of the sled?
	

	         2. Click Reset, and change the Red sled mass to 200 kg. Click Play. What is the speed of the sled at the bottom now?
	

	         3. Try other sled masses. Does the mass of the sled affect its final speed?
	

	

In this Gizmo, there is no friction, so there is no force slowing the sled down. As long as there is no friction, the sled’s final speed only depends on the starting height. In the real world, friction will affect the sled’s final speed.


         3. Explore: Click Reset. Use the Gizmo to measure the sled’s final speed when it starts at a height of 10 meters and when it starts at a height of 40 meters. Record these speeds below.


Speed when starting from 10 m: 
	

	Speed when starting from 40 m:
	

	

How does increasing the starting height affect the final speed? 




	                        


Activity B: 


Energy
	Get the Gizmo ready: 
         * Click Reset, and check that One sled is selected. 
         * Drag the red sled so that its height is 25.0 m. 
         * Set the Red sled mass to 100 kg.
	 SledWarsSE3 

	

Introduction: Energy can be found in several forms. Kinetic energy is energy of motion. Potential energy is stored energy, such as the stored energy of a sled at the top of a hill. 


Question: How are potential and kinetic energy related to one another?


         1. Collect data: Turn on Show energy. Notice that energy is measured in joules (J). In the table, record the potential and kinetic energy of the sled at the top of the slope, about halfway down, and at the bottom. (Use the Pause button to stop the sled halfway down.)


Location
	Height
	Potential energy
	Kinetic energy
	Total energy
	Top
	25.0 m
	

	

	

	Middle
	

	

	

	

	Bottom
	0.0 m
	

	

	

	

         1. How do the potential and kinetic energy change as the sled moves down the slope?




	

         2. Add the potential and kinetic energy for each location to find the total energy. What do you notice about the total energy? According to the law of conservation of energy, energy can be changed from one form to another, but it cannot be created or destroyed. That means that the total amount of energy at the beginning is the same as the total amount at the end.




	

         2. Explore: Vary the sled’s height and mass. Observe the effect of each change on the potential energy of the sled.
         1. How does potential energy change when height is increased? 
	

	         2. How does potential energy change when mass is increased?
	

	         3. Compare a sled’s potential energy at 10 m to its potential energy at 20 m. How does doubling height affect potential energy? 
	

	         4. Compare the potential energy of a 100-kg sled and a 200-kg sled at the same height. How does doubling mass affect potential energy? 
	

	





         3. Measure: Click Reset. Set the Red sled mass to 100 kg and the Snowmen to 1. Drag the sled to the top of the hill. Click Play and then Pause just before the sled hits the snowman.


         1. What is the speed of the sled before it hits the snowman?
	

	         2. What is the kinetic energy of the sled before it hits the snowman?
	

	         3. Click Play. What is the speed and kinetic energy of the sled after hitting the snowman?
	Speed:
Kinetic energy:
	         4. How much energy was used to destroy one snowman?
	

	

         4. Predict: Click Reset. Drag the sled to a height of 40 meters and set its mass to 160 kg.


         1. What is the potential energy of the sled?
	

	         2. How many snowmen do you think this sled can destroy? Explain your answer. 
	

	         3. Set the snowman to 5 and click Play. How many snowmen were destroyed.
	

	

         5. Extension 1: The formula for potential energy is PE = m • g • h. In this formula, PE stands for potential energy (in joules), m stands for mass (in kilograms), h stands for height (in meters), and g stands for acceleration caused by gravity. On Earth’s surface, g is 9.8 m/s2. This means that, every second, the speed of a falling object increases by 9.8 m/s.   


Calculate the potential energy of a 20-kg sled at 40 meters. ✏️Show your work in the space to the right. When you are done, use the Gizmo to check your answer.


Potential energy: 
	

	            6. Extension 2: The formula for kinetic energy is KE =   m • v2. In this formula, KE stands for kinetic energy (in joules), m stands for mass (in kilograms), and v2 stands for speed squared, or the speed multiplied by itself.   


Calculate the kinetic energy of a 20-kg sled moving 28.0 m/s. ✏️Show your work in the space to the right. To check your answer, set the height to 40 m and the mass to 20 kg. Click Play and wait until the sled gets to the bottom.
         
Kinetic energy:
	

	        




Activity C: 


Sled wars!
	Get the Gizmo ready: 
               * Select Two sleds. Check that Show energy is on.
               * Set the Red sled mass to 150 kg and its height to 20 m. Set the Blue sled mass to 100 kg and its height to 35 m.
	 SledWarsSE4 

	

Introduction: Burt the Yeti and his friend Karen are having a contest called a “sled war.” The winner is the sled that pushes the other sled backward when the two sleds collide.


Question: Who will win the sled war?


               1. Predict: Burt’s red sled is a bit heavier than Karen’s blue sled, but she is higher on the slope. 


               1. What is the potential energy of each sled?
	Red sled:
Blue sled:
	               2. Which sled do you think will push the other back when they collide? Explain.
	

	

               2. Test: Check that Sleds stick together is selected. Click Play. Click Pause after the two sleds collide with each other.


               1. Which sled won the sled war? Did this surprise you? 
	

	               2. After the collision, what is the kinetic energy of each sled? 
	

	               3. What happened to the combined energy of the two sleds when they collided? 
	

	               4. Because energy is conserved, the “lost” energy has actually been changed into other forms. Looking at the two sleds, what effect did some of this energy cause?
	

	

               3. Calculate: Click Reset. The momentum (p) of an object is the product of its mass and speed: p = m • v. The units of momentum are kilograms-meters per second, or kg•m/s. Click Play, and then click Pause just before the collision. 


A. What is the speed of the red sled?        
	

	B. What is the momentum of the red sled?
	

	C. What is the speed of the blue sled?
	

	D. What is the momentum of the blue sled?
	

	







               4. Gather data: Click Reset. For each combination of sled masses and starting heights below, click Play, and then click Pause just before the sleds collide. Record the speed and momentum of each sled. Then, click Play again and record the winning sled.


Trial
	1
	2
	3
	Red sled height
	50.0 m
	25.0 m
	5.0 m
	Red sled mass
	20 kg
	120 kg
	100 kg
	Red sled speed
	

	

	

	Red sled momentum
	

	

	

	Blue sled height
	10.0 m
	20.0 m
	45.0 m
	Blue sled mass
	100 kg
	130 kg
	40 kg
	Blue sled speed
	

	

	

	Blue sled momentum
	

	

	

	Winning sled
	

	

	

	

               1. In each trial, circle the greater momentum value. What is true about the momentum of the winning sled in each contest? 
	

	               2. If you knew each sled’s mass and speed, how could you determine which sled will be pushed backward in a collision?
	

	

               5. Explore: Click Reset. Select Sleds bounce. In this collision, large springs cause the sleds to bounce off each other. Using the same settings from the last trial above, click Play. 
What do you see?                                                                                 




	

               6. Extension: Click Reset. Without changing any of the sled masses or heights, record the potential and kinetic energy of each sled and find the total energy before the collision. Click Play and then Pause after the collision. Record the potential and kinetic energy of each sled after the collision, and find the total energy.




	Red PE
	Red KE
	Blue PE
	Blue KE
	Total energy
	Before collision
	

	

	

	

	

	After collision
	

	

	

	

	

	

Is energy conserved in this type of collision? Explain.